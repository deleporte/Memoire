%%%%%%%%%%%%%%%%%%%%%%%%%%%%%%%%%%%%
% Math style configuration page - v2
% By pavel. Last edited June 11 2015
% Tribute to Igor Kortchemski
%%%%%%%%%%%%%%%%%%%%%%%%%%%%%%%%%%%%

\usepackage{amsthm}
\usepackage{amssymb}
\usepackage{xparse}
%shortcuts
\newcommand{\R}{\mathbb{R}}
\newcommand{\C}{\mathbb{C}}
\newcommand{\Z}{\mathbb{Z}}
\newcommand{\N}{\mathbb{N}}
\newcommand{\Q}{\mathbb{Q}}
\newcommand{\fii}{\varphi}
\newcommand{\In}{\mathcal{I}}
\newcommand{\surj}{\twoheadrightarrow}
\newcommand{\inj}{\hookrightarrow}
\newcommand{\bij}{\leftrightarrow}
\newcommand{\dd}{\mathrm{d}}
\newcommand{\db}{\overline{\partial}_b}
\newcommand{\eitheta}{e^{i\theta}}

%operators
\DeclareMathOperator{\Hess}{Hess}
\DeclareMathOperator{\Vois}{Vois}
\DeclareMathOperator{\supp}{supp}
\DeclareMathOperator{\diag}{diag}



% theorems configuration

\newtheoremstyle{indented}{7pt}{7pt}{\addtolength{\leftskip}{2.5em}}{}{\bfseries}{.}{.5em}{}

\makeatletter
\newcounter{exo}[subsection]
\newenvironment{exo}[1][]
{\edef\@exer{#1}
\ifx\@exer\empty{}\elst{\setcounter{exo}{#1}\addtocounter{exo}{-1}}\fi\refstepcounter{exo}
\noindent{\bf Exercice \theexo.}
\hspace{0,5em}}
{\medskip}

\newcounter{sol}[subsection]
\newenvironment{sol}[1][]
{\edef\@exer{#1}
\ifx\@exer\empty{}\else{\setcounter{sol}{#1}\addtocounter{sol}{-1}}\fi
\refstepcounter{sol}
\noindent {\it \underline{Solution de l'exercice \thesol.}
\hspace{0.5em}}}
{\medskip}
\makeatother

\theoremstyle{indented}

\newtheorem{defn}{Définition}[chapter]

\newtheorem{theoremaux}{Théorème}   %theorems have their own numerotation and are accessible through the TOC
\NewDocumentEnvironment{theorem}{o}  %TOC accessibility
  {\IfNoValueTF{#1}
    {\theoremaux\addcontentsline{toc}{subsection}{\protect Théorème \thetheoremaux}}
    {\theoremaux[#1]\addcontentsline{toc}{subsection}{\protect\kern -1em Théorème \thetheoremaux\enspace(#1)}}
    \ignorespaces}
  {\endtheoremaux}
\newenvironment{preuve}{
\noindent \textbf{Démonstration. }}{\hfill $\square$}

\newtheorem{exemple}[defn]{Exemple}
\newtheorem{prop}[defn]{Proposition}
\newtheorem{corr}[defn]{Corollaire}
\newtheorem{por}[defn]{Porisme}
\newtheorem{ex}[defn]{Exemple}
\newtheorem{lem}[defn]{Lemme}
\newtheorem{ax}{Axiome}  %Axioms have their own numerotation

\theoremstyle{definition}
\newtheorem{rem}[defn]{Remarque} %remarks are not indented
\newtheorem{rems}[defn]{Remarques}


%--------------
% Mise en page mathématique
%--------------
\addtolength{\jot}{.2em}