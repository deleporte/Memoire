\chapter{Opérateurs Intégraux de Fourier}

\section{Intégrales à phases}

\subsection{Le lemme de phase non stationnaire}
Dans cette section, on établit un lemme d'une importance capitale dans la
construction et l'étude des intégrales dont l'un des termes oscille très
rapidement. On veut étudier le comportement asymptotique, quand $t \to
+\infty$, des intégrales du type
\begin{equation*}
  \int u e^{it\fii},
\end{equation*}
\noindent où les fonctions $u$ et $\fii$ sont lisses, et où évidemment $\Im
\fii \geq 0$.

Le lemme suivant montre que les seuls points intéressants sont ceux où
$d\fii=0$ et où $\Im \fii =0$ :
\begin{lem}[Phase non stationnaire]
\label{lem:phasenonstat}
Soit $K$ un compact de $\R^n$, ainsi que $X$ un voisinage de $K$, et soit $j$ et
$k$ des entiers naturels. Soit $\fii \in C^{k+1}(X)$
telle que $\Im \fii \geq 0$. Alors il existe une constante $C$, telle que
pour tout $t>0$ et $u \in C^k_0(K)$ :
\begin{multline*}
  t^{j+k}\left|\int e^{it\fii(x)}u(x)(\Im \fii(x))^j dx\right| \\\leq C
    \sum_{|\alpha| \leq k} \underset{K}{\sup}\left[|D^{\alpha}u|\,\left(|\fii'|^2 + \Im \fii\right)^{|\alpha|/2-k}\right]
\end{multline*}

\noindent De plus $C$ est bornée lorsque $\fii$ varie dans un domaine borné de $C^{k+1}(X)$.
\end{lem}

\subsubsection{Preuve du lemme de phase non-stationnaire}

On suit ici la preuve de Hörmander, exposée dans \cite{hormander2003analysis}.

  On introduit, pour des besoins de lisibilité, la notation suivante. Lorsque $f \in C^l$, on note, pour $x$ dans le domaine de définition de $f$ :
  \begin{equation*}
    |f|_l(x) := \sum_{|\alpha|=l}|\partial^{\alpha}f|(x)
  \end{equation*}
  
  Lorsque $k=0$, l'inégalité à démontrer devient :
\begin{equation*}
  \left|\int_K u(x)t^{j}(\Im \fii(x))^j e^{it\fii(x)}dx\right| \stackrel{?}{\leq} 
    C(\fii) \sup |u|.
\end{equation*}
\noindent Or la fonction $t \mapsto t^je^{-t}$ est bornée sur $\R^+$, donc en fait $C$ ne dépend même pas de $\fii$.

Pour prouver le cas général par récurrence sur $k$, on raisonne par intégrations par parties. Posons $N=|\fii'|^2+\Im\fii$. En remplaçant $\fii$ par $\fii+i\epsilon$, on peut supposer que $\Im N > 0$, sous réserve de montrer que $C$ ne dépend pas de $\epsilon$.

Alors on a :
\begin{align*}
  1 &= \cfrac{N}{N}\\
  &= \cfrac{\Im \fii}{N} + \sum_{\nu=1}^{n}\cfrac{(\partial_{\nu}
    \overline{\fii})(\partial_{\nu}
    \fii)}{N}
\end{align*}

\noindent En suivant cette décomposition, on définit 
\begin{align*}
u_0&=\cfrac{u}{N}\\
u_{\nu} &= \cfrac{u\,\partial_{\nu}\overline{\fii}}{N},
\end{align*}
\noindent et alors :
    \begin{equation*}
      u = \sum_{\nu = 1}^{n}u_{\nu}\partial_{\nu} \fii + u_0 \Im \fii
    \end{equation*}

\noindent Cette écriture est motivée par le fait que $it(\partial_{\nu}
\fii)e^{it\fii} = \partial_{\nu} e^{it\fii}$.

On a donc, en remplaçant $u$ et en effectuant une intégration par parties :
\begin{multline*}
  \int u\,(\Im\fii)^je^{it\fii} = \int u_0\,(\Im\fii)^{j+1}e^{it\fii} \\+
  \cfrac{i}{t} \sum_{\nu=1}^{n}\left[\int (\partial_{\nu}u_{\nu})(\Im
    \fii)^je^{it\fii} + j\int u_{\nu}\,(\partial_{\nu}\Im \fii)(\Im \fii)^{j-1}e^{it\fii}\right].
\end{multline*}

Commençons par traiter le cas $j=0$. On a alors, par l'hypothèse de récurrence :
\begin{multline*}
  t^{j+k}\left|\int u\,(\Im \fii)^je^{it\fii} \right| \leq
  C\left[\sum_{|\alpha| \leq k-1}\sup |D^{\alpha}u_0|N^{|\alpha|/2-k+1}\right. \\+
    \left.\sum_{1 \leq |\alpha| \leq k}\sup
    |D^{\alpha}u_{\nu}|N^{|\alpha|/2-k+1}\right] 
\end{multline*}

\noindent Où $C$ est bornée lorsque $f$ varie dans un domaine borné.

En remplaçant $u_0$ et $u_{\nu}$ par leur valeur, il ne reste plus qu'à montrer que les puissances se compensent, c'est-à-dire, pour tout $l$ :
\begin{equation*}
  \sqrt{N}|u_{\nu}|_l + \sqrt{N}|u_0|_l \stackrel{?}{\leq} C \sum_{|\alpha| \leq l}|D^{\alpha}u|N^{(\alpha|-l)/2}.
\end{equation*}

Raisonnons par récurrence sur $l$. Pour $l=0$, d'abord, on a  $\sqrt{N} |u/N \partial_{\nu}\fii| + N |u/N| \leq (\sqrt{N}+1)|u|$. Pour $l$ non nul, si $|\alpha|=l$, alors en appliquant $\partial^{\alpha}$ à l'équation $Nu_{\nu}=u \partial_{\nu}\overline{\fii}$, et en contrôlant toutes les dérivées d'ordre plus grand que $2$ de $N$ et de $\fii$ par des constantes (qui ne dépendent que de $\|fii\|_{W^{l,\infty}(X)}$), on a :
\begin{equation*}
  N|u_{\nu}|_l \leq C\left(|N|_1|u_{\nu}|_{l-1} + \sum_{l'=0}^{l-2}|u_{\nu}|_{l'} + |u|_l|\fii|_1 + \sum_{l'=0}^{l-1}|u|_{l'}\right)
\end{equation*}

Pour évaluer $|N|_1$ et $|\fii|_1$, on se sert du :
\begin{lem}
  \label{horm7.7.2}
  Si $g \in C^2(]-1,1[)$ vérifie $g \geq 0$, alors on a :
  \begin{equation*}
    |g'(0)|^2 \leq g(0)\left(g(0) + 2 \sup |g''|\right)
  \end{equation*}
\end{lem}
\begin{proof} (du lemme) En notant $M= \sup |g''|$ on a, pour tout $|x| \leq 1$, que $g(0)+xg'(0)+Mx^2/2 \geq g(x) \geq 0$, et l'étude des racines de ce polynôme du second degré nous fournit le résultat demandé.\end{proof}

En majorant à nouveau $|N|_2$ et $|\fii|_2$ par des constantes qui restent bornées lorsque $\fii$ varie dans un domaine borné de $C^{l+1}$, on trouve que $|N|_1 \leq C N^{1/2}$, et la même propriété pour $\fii$. Alors le développement précédent devient :
\begin{equation*}
  N|u_{\nu}|_l \leq C\left(N^{1/2}|u_{\nu}|_{l-1} + \sum_{r=0}^{l-2}|u_{\nu}|_{r} + |u|_l|\fii|^{1/2} + \sum_{r=0}^{l-1}|u|_{r}\right)
\end{equation*}

\noindent Appliquons l'hypothèse de récurrence aux deux premiers termes :
\begin{align*}
N^{1/2}|u_{\nu}|_{l-1} &\leq C \sum_{r=0}^{l-1} |u|_r N^{(r-l+1)/2}\\
|u_{\nu}|_r & \leq C\sum_{s=0}^{r}|u|_sN^{(s-r-1)/2}\\
  &\leq C\sum_{s=0}^{n-2}|u|_sN^{(s-l+1)/2}
\end{align*}

\noindent En injectant ces résultats et en divisant par $N^{1/2}$, on obtient le résultat désiré. Pour $u_0$, la méthode est la même, mais il n'y a pas de termes en $\fii$.

Ceci conclut le cas $j=0$. Pour les autres valeurs de $j$, on raisonne par récurrence. Si on suppose que le résultat est vrai pour $(k,j-1)$, alors on peut traiter le dernier terme de la décomposition, qui est contrôlé par :
\begin{equation*}
  C \sum_{\nu=1}^{n} \sum_{|\alpha| \leq
    k}\sup|D^{\alpha}(u_{\nu}\partial_{\nu}\Im \fii)| N^{|\alpha|/2-k}
\end{equation*}
\noindent On peut alors développer $|u_{\nu}\partial_{\nu}\Im \fii|_l$ pour le contrôler en utilisant à nouveau le lemme \ref{horm7.7.2}, exactement comme précédemment.

Ceci conclut la preuve du lemme \ref{lem:phasenonstat}.

\subsection{Le théorème de phase stationnaire}

Nous nous penchons ici sur le comportement des intégrales oscillantes, mais cette fois-ci, au voisinage des points $x_0$ qui vérifient $N=0$, donc $\Im \fii(x_0) = 0$ et $d\fii(x_0)=0$. On appelle \emph{point critique} un tel point $x_0$. On suppose toujours ici que $\fii$ est \emph{positive}, c'est-à-dire vérifie $\Im \fii \geq 0$, et que tous les points critiques de $\fii$ sont \emph{non-dégénérés}, et on entend par là que $\Hess \fii$ est inversible.
L'ensemble des points critiques est alors discret (et son intersection avec $K$ est finie), ce qui permet de raisonner par partition de l'unité.

Remarquons que les deux conditions de positivité et de non dégénérescence impliquent que  $Sp(i\Hess\fii(x_0)) \subset \{z \in \C, \Re z \leq 0\}$. Sur ce demi-plan, il existe une racine carrée. On choisit celle qui est à valeurs dans $\Im \lambda \geq 0$, ce qui assure un sens univoque à l'écriture $\sqrt{\det i\Hess \fii (0)}$.

 On peut alors énoncer le 
\begin{theorem}[de phase stationnaire]\label{th:phasestat}Soit $\fii \in C^{2j+1}(\Vois(K))$ telle que a $0$ pour seul point critique dans le compact $K \subset \R^n$.
Il existe alors une famille $(C_l)_{0 \leq l \leq j}$, d'opérateurs différentiels de degré au plus $2l$, tels que, pour toute fonction $u\in C_c^{2j+1}(K)$, uniformément en $t$, on ait :
\begin{equation*}
  \int_Ke^{it\fii}u = t^{-n/2}\sum_{l=0}^{j}t^{-l}C_lu(0) + O\left(t^{-(n+1)/2-j}\|u\|_{W^{2j+n,\infty}}\right).
\end{equation*}

En particulier on a $C_0 = \cfrac{\pi^{n/2}}{\sqrt{\det i\Hess{\fii}(0)}}$.
\end{theorem}

\subsubsection{Preuve dans le cas d'une phase quadratique}
Nous commençons par traiter le cas où $\fii(x)=iq(y,y)$, où $q$ est une forme bilinéaire définie positive ; $\fii$ est alors une phase quadratique imaginaire pure.

Quitte à faire un changement de variables par une matrice orthogonale, on peut supposer que 
\begin{equation*}
  q(y,y)=\sum_{\nu}\lambda_{\nu}\cfrac{y_{\nu}^2}{2}
\end{equation*}

Soit maintenant $u\in C_c^{2j+1}(K)$. On dispose, sur tout $\R^n$, de l'inégalité de Taylor-Lagrange :
\begin{align*}
  u(x)&=u(0) + \sum_{k=1}^{2j}U_k(x)+R(x)\\
  |R(x)| &\leq \sup(|u|_{2j+1})|x|^{2k+1}
\end{align*}

\noindent où chaque $U_k$ est $k$-multilinéaire. On peut alors faire le changement de variable $y = x/\sqrt{t}$, et remplacer $u$ par le développement précédent. Par parité, les $U_k$ pour $k$ impairs ne vont pas contribuer à l'intégrale. Ainsi,
\begin{multline*}
  \int_{\R^n}e^{it\fii}u = t^{-n/2}\sum_{l=0}^{j}t^{-l}\int e^{-q(y,y)}U_{2l}(y)dy \\+ O\left(t^{-(n+1)/2-j}\sup |u|_{2j+1}\right).
\end{multline*}

Les $U_{2l}$ ne dépendent que des dérivées d'ordre $2l$ de $u$ en $0$, donc on a bien la forme demandée. Toutefois, on va établir explicitement quels sont les opérateurs différentiels qui entrent en jeu.

Par intégrations par parties successives, on sait que 
\begin{equation*}
  \int_{\R} e^{-\lambda \frac{x^2}{2}}x^{2l}dx = \sqrt{\cfrac{\pi}{\lambda^{2l+1}}}\cfrac{(2l)!}{2^l l!}.
\end{equation*}
\noindent Par ailleurs, on peut expliciter les fonctions $U_l$ par la formule de Taylor :
\begin{equation*}
  U_{2l}(x_1,\ldots,x_n) = \sum_{|\alpha|=2l}\partial^{\alpha}u(0)\prod_{\nu=1}^{n}\cfrac{x_{\nu}^{\alpha_{\nu}}}{\alpha_{\nu}!}
\end{equation*}

Seuls les $\alpha$ dont tous les coefficients sont pairs vont contribuer à l'intégrale. En conjuguant par la transformée de Fourier, on peut donc écrire :
\begin{equation*}
  \int e^{-q(y,y)}U_{2l}(y)dy = \sqrt{\cfrac{\pi^n}{\prod \lambda_{\nu}}}\sum_{|\alpha|=l}\int (i\xi)^{2\alpha}\hat{u}(\xi)d\xi \prod_{\nu=1}^{n} \cfrac{1}{(\lambda_{\nu})^{\alpha_{\nu}}\alpha_{\nu}!}
  \end{equation*}
  
On va synthétiser ce développement à l'aide de deux notations. On note $P_l$ le projecteur qui à une fonction de classe $C^{\infty}$ associe les termes d'ordre $l$ de son développement de Taylor, on a donc $U_{2l}=P_{2l}u$. Par ailleurs, si $Q= \sum Q_{\alpha}X^{\alpha}$ est un polynôme, on note $Q(D)$ l'opérateur différentiel $\sum Q_{\alpha}(i\partial)^\alpha$, et alors on a :
\begin{equation}
\label{eq:phasequad}
  \int e^{-q(y,y)}U_{2l}(y)dy =\cfrac{\pi^{n/2}}{\sqrt{\det \Hess u(0)}}\,P_j\left(e^{-\frac{1}{2}q^{-1}(D,D)}\right)u(0)
  \end{equation}

\noindent Nous avons donc une expression explicite de $C_l$. On constate qu'ici, les $C_l$ ne font intervenir que des dérivées d'ordre exactement $l$.

Nous concluons par une remarque importante.

\begin{rem}
Les expressions à gauche et à droite dans l'équation (\ref{eq:phasequad}) sont analytiques par rapport à $q$, et admettent chacune une unique extension holomorphe dans l'ensemble des formes quadratiques complexes à partie réelle strictement positive. A droite, cela est dû aux considérations sur la racine du déterminant que nous avons déjà évoquées. A gauche, l'ajout d'une partie réelle ne change pas la convergence de l'intégrale et on peut utiliser le théorème de convergence dominée.

Tout ceci a pour conséquence que lesdites extentions holomorphes sont égales. On vient donc de démontrer le lemme de phase quadratique pour des phases complexes dont la partie imaginaire pure est non dégénérée.
\end{rem}

Il reste à examiner le cas où la $\Im \fii$ est seulement positive (et pas définie positive comme précédemment). Seule l'intégrale sur $ker \Im \fii$ nous intéresse, puisque l'on peut calculer l'intégrale sur le complémentaire orthogonal par la méthode précédente, et que l'on dispose alors de bonnes conditions de domination pour assurer un comportement régulier de cette intégrale par rapport au paramètre dans $ker \Im \fii$. Autrement dit, l'examen des phases quadratiques réelles suffit.

Notre contrôle est ici moindre que dans le cas précédent. En effet, la suite des fonctions définies par
\begin{equation*}
  u_n(x)=\chi(x)e^{-inq(x,x)}
\end{equation*}
est un contre-exemple à la proposition :
\begin{equation*}
  \forall (K,\fii) \exists (C,t_0) \forall u \in \C^{\infty}_c(K),t>t_0\: \left|\int_Ke^{it\fii}u\right| \leq Ct^{-n/2}\|u\|_{\infty}
\end{equation*}

On considère donc $\fii(x)=q(x,x)$ une phase réelle, avec $\det \fii \neq 0$. La première étape consiste à calculer la transformée de Fourier de la fonction $x\mapsto e^{itq(x,x)}$. Pour cela on se ramène, par un changement de variables orthogonal, au cas où $q$ est diagonale ; on peut alors séparer les variables. Il n'y a plus qu'à reporter le résultat bien connu :
\begin{equation*}
  \mathcal{F}\left(x\mapsto e^{i\frac{x^2}{2}}\right) : \cfrac{\sqrt{\pi}}{\sqrt{i}}\xi \mapsto e^{i\frac{\xi^2}{2}}
\end{equation*}
\noindent Et on obtient alors :
\begin{equation*}
  \mathcal{F}\left(x\mapsto e^{itq(x,x)}\right) : \xi \mapsto \cfrac{\pi^{n/2}}{\sqrt{t^n \det (iq)}}e^{i\frac{q^{-1}(x,x)}{4t}}
\end{equation*}
La seconde étape est d'appliquer ce calcul à notre problème. Pour $u \in C^{\infty}_c(\R^n)$, la fonction $f_t=e^{it\fii}u$ est en particulier dans la classe de Schwartz, donc on peut écrire
\begin{equation*}
  \int_{\R^d}e^{it\fii}u = \widehat{f_t}(0)
\end{equation*}

Soit un entier $j \in \N$. Pour tout $t$ fixé, on peut développer l'exponentielle en série entière, et écrire de manière licite :
\begin{multline*}
  \widehat{f_t}(0)=t^{-n/2}\cfrac{\pi^{n/2}}{\sqrt{\det{iq}}}\left[\sum_{k=0}^{j}\cfrac{t^{-k}}{k!}\left(\cfrac{(-1)^n}{4}q^{-1}(D,D)\right)^ku(0)\right. \\+ \left.t^{-j-1}\sum_{k=0}^{+\infty}\cfrac{t^k}{(k+j+1)!}\int (iq(x,x)/4)^k(iq(x,x)/4)^{j+1}u(x)dx\right]
\end{multline*}

On voit apparaître les opérateurs différentiels que l'on recherche :
\begin{equation}
\label{eq:coeffquad}
\cfrac{(-1)^{nk}\pi^{n/2}}{4^kk!\sqrt{\det{iq}}}\left(q^{-1}(D,D)\right)^k
\end{equation}

Les autres termes peuvent être sommés, et on a un contrôle dans l'espace de Sobolev $H^{-n/2}$ :

\begin{align*}
	\left|t^{-j-1}\sum_{k=0}^{+\infty}\cfrac{t^k}{(k+j+1)!}\int (iq/4)^k(iq/4)^{j+1}u\right| \leq C(j,q)\|u\|_{W^{2,j+1+n/2}}\\
	\leq C(j,q,K)\|u\|_{W^{\infty,j+1+n/2}}
\end{align*}

Ce qui conclut la preuve.

\subsubsection{Preuve affaiblie dans le cas général}
À partir du lemme de phase quadratique, il existe deux manières de démontrer le cas général. La première est d'utiliser le lemme de Morse pour se ramener, après un changement de variable, à une phase quadratique. Cette méthode présente néanmoins de nombreuses difficultés supplémentaires dans le cas d'une phase complexe puisqu'au lieu d'intégrer le long de l'axe réel, on intègre maintenant sur une courbe tangente à l'axe réel en l'origine. Une preuve détaillée est explicitée dans \cite{melin1975fourier}.

La seconde méthode est d'isoler, parmi la phase, le terme quadratique du reste, et c'est cette méthode que nous allons détailler. 

Nous allons démontrer un résultat plus faible que le théorème originel, où l'on contrôle par un nombre beaucoup plus important de dérivées.

\begin{prop}[Phase stationnaire faible]\label{prop:phasestat}On suppose que $\fii \in C^{2j+1}(\Vois(K))$ a $0$ pour seul point critique dans le compact $K \subset \R^n$.
Il existe alors une famille $(C_l)_{0 \leq l \leq j}$, d'opérateurs différentiels de degré au plus $2l$, tels que, pour toute fonction $u\in C_c^{2j+1}(K)$, lorsque $t\to +\infty$, on ait :
\begin{equation*}
  \int_Ke^{it\fii}u = t^{-n/2}\sum_{l=0}^{j}t^{-l}C_lu(0) + O\left(t^{-(n+1)/2-j}\|u\|_{W^{6j+1+n/2,\infty}}\right).
\end{equation*}

En particulier on a $C_0 = \cfrac{\pi^{n/2}}{\sqrt{\det i\Hess{\fii}(0)}}$.
\end{prop}
\begin{preuve}
Soit $\fii \in \C^{j+1}(\Vois(K))$ une phase vérifiant $\Im \fii \geq 0$, et dont $0$ est le seul point critique non dégénéré. On décompose :
\begin{equation*}
\fii(x) = iq(x)+r(x),
\end{equation*}
\noindent où $q$ est une forme quadratique bien particulière. Comme dans le cas précédent, on va pouvoir faire une disjonction de cas.
\begin{enumerate} \item Ou bien $\Re q >>0$ ;
 \item Ou bien $\Re q \geq 0$ et $\Im q$ est non dégénérée. 
\end{enumerate}
Dans tous les cas, $r$ vérifie :
\begin{equation*}
\forall |\alpha| \leq 2,\, \partial^{\alpha}r(0)=0.
\end{equation*}

L'utilisation directe du lemme de phase quadratique ne permet pas de conclure car on a besoin de savoir que $\Im \fii$ ne s'annule pas ailleurs qu'en 0. On va donc découper l'intégrale à estimer en une intégrale sur $\Vois(K) \cap B(0,t^{-1/3})$ et une intégrale sur $\Vois(K) \setminus B(0,t^{-1/3})$.

Le second morceau est le plus simple à estimer. On sait qu'il existe $\epsilon$ tel que sur $B(0,\epsilon)$, on ait $|\fii-iq|\leq \frac 14 q$. En particulier, il existe une constante $C$ telle que, pour $t$ grand, sur $B(0,\epsilon)\setminus B(0,t^{-1/3})$, on ait :
\begin{equation*}
  \Im\fii + |\fii'|^2 > Ct^{-2/3}.
\end{equation*}

Soit alors $j\in \N$, on peut estimer :
\begin{align*}
&\int_{B(0,\epsilon) \setminus B(0,t^{-1/3})}e^{i\fii(x)}u(x)dx \\
&\leq C t^{3(-2j-1)}\sum_{|\alpha| \leq 3(j+1)}\sup|D^{\alpha}u|(\Im \fii + |\fii'|^2)^{|\alpha|/2-3(2j-1)}\\
&\leq Ct^{-(2j+1)}\|u\|_{W^{3(j+1),\infty}}
\end{align*}
Ici, on constate déjà une perte de dérivées.

Sur $\Vois(K)\setminus B(0,\epsilon)$, on a $\Im \fii + |\fii|^2 > C$, donc le lemme de phase non stationnaire s'applique directement et on a :
\begin{equation*}
  \int_{\Vois(K)\setminus B(0,\epsilon)}e^{it\fii}u \leq Ct^{-j-(n+1)/2}\|u\|_{W^{j+(n+1)/2,\infty}}.
\end{equation*}

Estimons à présent le premier morceau. On réalise d'abord un développement de $e^{itr}$ en série entière en $t$ ; l'interversion suivante est licite pour tout $t$:
\begin{equation*}
  \int_{B(0,t^{-1/3})}e^{it\fii}u = \sum_{k=0}^{+\infty}\cfrac{t^k}{k!}\int_{B(0,t^{-1/3})}e^{-tq(x)}\left(ir(x)\right)^ku(x)dx.
\end{equation*}

Traitons chacune de ces intégrales séparément, de la même manière que dans le cas quadratique. On décompose d'abord :
\begin{equation*}
  \left(ir(x)\right)^ku(x) = \sum_{l=3k}^{2j}U_{l,k}(x) + R_k(x).
\end{equation*}
\noindent Ici, chaque $U_{l,k}$ est homogène de degré $j$, et s'exprime uniquement en fonction de $\fii$ et des dérivées de $u$ en $0$, et on a de plus
\begin{equation*}
  |R(x)| \leq C(j) \sup (|r^ku|_{2j+1}) x^{2j+1}
\end{equation*}

On réalise alors un changement de variable $y=x\sqrt{t}$, pour arriver à la forme suivante :
\begin{multline*}
  \int_{B(0,t^{-1/3})}e^{-tq(x)}\left(ir(x)\right)^ku(x)dx \\= t^{-n/2}\sum_{l=3k}^{2j}\left(t^{-l/2}\int_{B(0,t^{1/6})}e^{-q(y)}U_{l,k}(y)dy\right. \\+ \left.\int_{B(0,t^{1/6})}e^{-q(y)}R_k(y/\sqrt{t})dy\right)
\end{multline*}

D'une part, on a, pour tous $k,l$, pour la même raison que précédemment :
\begin{equation*}
  \int_{\R^n \setminus B(0,t^{1/6})}e^{-q(y)}U_{l,k}(y)dy = O\left(t^{-j-(n+1)/2}|r^ku|_{l}(0)\right)
\end{equation*}
\noindent On peut donc remplacer sans problème un nombre fini de termes :
\begin{multline*}
\int_{B(0,t^{-1/3})}e^{-tq(x)}u(x)dx \\= t^{-n/2}\sum_{k=0}^{+\infty}\cfrac {t^k}{k!}\left(\sum_{l=3k}^{2j} t^{-l/2}\int_{\R^n}e^{-q(y)}U_{l,k}(y)dy\right. \\+ \left.\int_{B(0,t^{1/6})}e^{-q(y)}R(y/\sqrt{t})dy\right) \\+ O\left(t^{-j-(n+1)/2}\|u\|_{W^{2j+1,\infty}}\right)
\end{multline*}
\noindent Après élimination des termes avec $l$ impairs, on peut regrouper ces intégrales, dont on connaît un développement exact en fonction des dérivées de $u$ en $0$. On constate que seules des puissances entières de $t$ apparaissent, et par ailleurs, dans le terme en $t^m$, on dérive $u$ au plus $2m$ fois. On peut donc écrire :
\begin{align*}
\int_{B(0,t^{-1/3})}e^{-tq(x)}u(x)dx& \\= t^{-n/2}\sum_{m=0}^{2j} & A_mu(0) \\+ t^{-n/2}\sum_{k=0}^{+\infty}&\cfrac{t^k}{k!}\int_{B(0,t^{1/6})}e^{-q(y)}  R(y/\sqrt{t})dy \\&+ O\left(t^{-j-(n+1)/2}\|u\|_{W^{2j+1,\infty}}\right)
\end{align*}

D'autre part, souvenons-nous maintenant que :
\begin{equation*}
  |R(x)| \leq C(j) \sup (|r^ku|_{2j+1}) |x|^{2j+1}
\end{equation*}

Traitons maintenant le cas 1. L'autre cas se traite par un argument de transformée de Fourier, similaire à la preuve du lemme de phase quadratique. 

Par une disjonction de cas, si $2j+1 \geq 3k$, alors
\begin{multline*}
  \left|\cfrac{t^k}{k!}\int_{B(0,t^{1/6})}e^{-q(y)}R(y/\sqrt{t})dy\right| \\\leq C(j)\|u\|_{W^{2j+1,\infty}}t^{-j-1/2}\int_{B(0,t^{1/6})}e^{-q(y)}|y|^{2j+1}dy
\end{multline*}
\noindent On constate qu'avec cette méthode, on perd des dérivées :
\begin{equation*}
  \left|\cfrac{t^k}{k!}\int_{B(0,t^{1/6})}e^{-q(y)}R(y/\sqrt{t})dy\right| \leq C(j)\|u\|_{W^{2j+1,\infty}}t^{-j/3-1/6}
\end{equation*}

De même, si $2j+1 < 3k$, alors on a $|r^ku|(x)_{2j+1} \leq Cx^{3k-2j-1}$, donc :
\begin{multline*}
  \left|\cfrac{t^k}{k!}\int_{B(0,t^{1/6})}e^{-q(y)}R(y/\sqrt{t})dy\right| \\\leq C(j)\|u\|_{W^{2j+1,\infty}}t^{1/3(2j+1-3k)}t^{-j-1/2}\int_{B(0,t^{1/6})}e^{-q(y)}|y|^{2j+1}dy
\end{multline*}
\noindent Et on a la même perte de dérivées :
\begin{equation*}
  \left|\cfrac{t^k}{k!}\int_{B(0,t^{1/6})}e^{-q(y)}R(y/\sqrt{t})dy\right| \leq \cfrac{C(j)}{k!}\|u\|_{W^{2j+1,\infty}}t^{-j/3-1/6}
\end{equation*}

Ce terme d'erreur est plus grand que certains des termes du développement. En éliminant ces termes et en remplaçant $j$ par $3j$, on conclut la démonstration de la proposition \ref{prop:phasestat}.
\end{preuve}

\begin{rem}
	Le théorème \ref{th:phasestat} est démontré dans \cite{hormander2003analysis}, en utilisant une décomposition de l'espace des fréquences.
\end{rem}

\subsubsection{Un exemple}
On va calculer les premiers coefficients $C_0, C_1, C_2$ associés à la phase réelle sur $\R$ définie par
\begin{equation*}
	\fii(x)=x^2 + x^3 + x^4.
\end{equation*}
\noindent Cette phase est strictement positive ailleurs qu'en $0$, où elle s'annulle exactement à l'ordre $2$, avec $\Hess \fii(0)=2$.

On a donc directement
\begin{equation*}
	C_0 = \sqrt{\cfrac{\pi}{2i}}
\end{equation*}

Commençons par calculer les premiers coefficients $(C_i')$ qui correspondent à la seule phase $x \mapsto x^2$, grâce à la formule (\ref{eq:coeffquad}) : 

\begin{align*}
	C_1' &= -\cfrac{1}{4}C_0\partial^2\\
	C_2' &= \cfrac{1}{32}C_0\partial^4\\
	C_3' &= -\cfrac{1}{384}C_0\partial^6\\
	C_4' &= \cfrac{1}{6144}C_0\partial^8\\
	C_5' &= -\cfrac{1}{122880}C_0\partial^{10}\\
	C_6' &= \cfrac{1}{2949120}C_0\partial^{12}
\end{align*}

On sait maintenant que le développement jusqu'à l'ordre 2 s'écrit :

\begin{multline*}
	C_0u(0) + t^{-1}C'_1(u) + t^{-2}C'_2(u + tru) + t^{-3}C'_3(u + tru + \frac{t^2}{2}r^2u) \\+ t^{-4}C'_4(u + tru + \frac{t^2}{2}r^2u) + t^{-5}C'_5(u + tru + \frac{t^2}{2}r^2u + \frac{t^3}{6}r^3u) \\+ t^{-6}C'_6(u + tru + \frac{t^2}{2}r^2u + \frac{t^3}{6}r^3u + \frac{t^4}{24}r^4u) \\+ t^{-7}C'_7(u + tru + \frac{t^2}{2}r^2u + \frac{t^3}{6}r^3u + \frac{t^4}{24}r^4u) \\+ \ldots
\end{multline*}

Ici, on tronque à chaque fois le développement de l'exponentielle pour ne garder que les termes qui contribuent, et on réalise le développement en phase stationnaire jusqu'à ce que la compensation des puissances de $t$ ne crée que des puissances supérieures à deux. Comme on le voit, en pratique, il est nécessaire d'aller loin dans le développement, rendant les calculs pénibles.

On peut ensuite estimer chaque terme, en ne gardant que les grandes puissances de $t$ :

\begin{align*}
	t^{-2}C'_2(u + tru)/C_0 &= \cfrac{t^{-2}}{32}u^{(4)}(0) + \cfrac{3t^{-1}}{16}u'(0) + \cfrac{3t^{-1}}{4}u(0)\\
	t^{-3}C'_3(tru + \frac{t^2}{2}r^2u)/C_0 &= -\cfrac{t^{-2}}{64}u^{(3)}(0) - \cfrac{t^{-2}}{16}u^{(2)}(0) - \cfrac{15t^{-1}}{16}u(0)\\
	t^{-4}C'_4(\frac{t^2}{2}r^2u)/C_0 &= \cfrac{15t^{-2}}{256}u^{(2)}(0) + \cfrac{105t^{-2}}{128}u'(0) + \cfrac{105t^{-2}}{32}u(0)\\
	t^{-5}C'_5(\frac{t^3}{6}r^3u)/C_0 &= -\cfrac{63t^{-2}}{128}u'(0) - \cfrac{945t^{-2}}{64}u(0)\\
	t^{-6}C'_6(\frac{t^4}{24}r^4u)/C_0  &= \cfrac{3465t^{-2}}{512}u(0)
\end{align*}

Finalement on peut conclure :
\begin{align*}
	C_1 &= \sqrt{\cfrac{\pi}{2i}}\left(-\cfrac{1}{4}\partial^2 + \cfrac{3}{16}\partial - \cfrac{3}{16}\right)\\
	C_2 &= \sqrt{\cfrac{\pi}{2i}}\left(\cfrac{1}{32}\partial^4 - \cfrac{1}{64}\partial^3 - \cfrac{1}{256}\partial^2 + \cfrac{21}{64}\partial - \cfrac{2415}{512}\right)
\end{align*}
\subsection{Classes d'intégrales à phases}
Soit $X$ un ouvert de $\R^n$ et $\Gamma$ un cône ouvert de $X \times (\R^d \setminus \{0\})$, on veut étudier des \emph{intégrales à phase} sur $\Gamma$. Pour cela on introduit les définitions suivante :

\begin{defn}Une fonction $\fii \in C^{\infty}(\Gamma)$ est appelée \emph{phase} lorsqu'elle vérifie les conditions suivantes :
\begin{enumerate}
  \item $\fii$ est 1-homogène par rapport à la seconde variable;
  \item Sur $\Gamma$, on a $\Im \fii \geq 0$;
  \item La forme $d\fii$ ne s'annulle pas.
\end{enumerate}
\end{defn}
Cette définition va nous permettre d'appliquer les outils techniques développés précédemment.

\begin{defn}[Classes de symboles] Soient $m\in \R$ et deux paramètres $\rho \in ]0,1],\,\delta \in [0,1[$. Pour $\Gamma$ un cône de $X\times \R^d$, l'espace $S_{\rho,\delta}^m(\Gamma)$ est défini comme le sous-ensemble des $a \in C^\infty(\Gamma)$ vérifiant que, pour pour tout $K \subset \subset X$, pour tous multi-indices $\alpha,\beta$, il existe une constante $C$ telle que sur tout $\Gamma \cap (K\times \R^d)$, on ait
\begin{equation*}
  |\partial^{\beta}_x\partial^{\alpha}_{\theta}a(x,\theta)| \leq C(1+|\theta|)^{m+\delta|\beta|-\rho|\alpha|}
\end{equation*}
Ce sous-ensemble forme bien un espace vectoriel, et les constantes optimales dans l'équation précédente en font un espace de Fréchet.

On notera $S^m(\Gamma)=S^m_{1,0}(\Gamma)$, et encore
\begin{align*}
  S_{\rho,\delta}^{+\infty}=\bigcup_{m \in \Z}S^m{\rho,\delta}\\
  S_{\rho,\delta}^{-\infty}=\bigcap_{m \in \Z}S^m{\rho,\delta}.
\end{align*}
\end{defn}
Les symboles sont donc des fonctions dont le comportement en l'infini en fonction de $\theta$ ressemble à celui des fractions rationnelles. Les espaces $S^m_{\rho,\delta}$ sont croissants avec $m$ et $\delta$ et décroissants avec $\rho$.

Par exemple, les fonctions $m$-positivement homogènes en $\theta$ sont dans $S^m$.

On peut alors adapter le lemme de Borel à cette situation :

\begin{lem}\label{lem:Borel}
	Soit $m \in \R$, $\rho\in ]0,1], \delta \in [0,1[$, et $(a_j)$ une suite de fonctions dans $C^{\infty}(\Gamma)$ telle que, pour tout $j$, $a_j \in S^{m-j}_{\rho, \delta}(\Gamma)$. Alors il existe $a\in S^{m}_{\rho, \delta}(\Gamma)$ tel que, pour tout entier $K$, on a $a - \sum_{j=0}^Ka_j \in S^{m-K-1}_{\rho, \delta}(\Gamma)$.
	
	La fonction $a$ est unique modulo $S^{-\infty}_{\rho, \delta}(\Gamma)$, et on écrit :
	
	\begin{equation}
	\label{eq:Borel}
	a \sim \sum_{j=0}^{+\infty}a_j.
	\end{equation}
\end{lem}

Ce lemme amène à la notion de symbole classique :

\begin{defn}
	On dit que $a$ est un symbole classique, et on écrit $a \in S^m_{clas}(\Gamma)$, lorsque $a$ s'écrit sous la forme (\ref{eq:Borel}) avec $a_j \in S^m(\Gamma)$ (donc $\rho = 1$, $\delta = 0$), et $a_j$ homogène de degré $m-j$ en $\xi$.
\end{defn}

Une autre caractéristique topologique des $S^m_{\rho, \delta}$ sera très utile : la densité de $C^{\infty}_c(\Gamma)$.

Ainsi, par le théorème de Hahn-Banach, pour définir une opération linéaire sur $S^m_{\rho, \delta}$, il suffit de la définir sur $C^{\infty}_c(\Gamma)$ puis de vérifier que la définition est continue par rapport à la topologie de $S^m_{\rho, \delta}(\Gamma)$.

On peut alors énoncer la proposition suivante :
\begin{prop}
Soit $\fii$ une phase sur le cône ouvert $\Gamma$ et $F$ un cône fermé inclus dans $\Gamma$. Soit $m \in \R,\rho \in ]0,1],\delta \in [0,1[$. Alors la fonctionnelle qui à $a \in C^{\infty}_c\left(\mathring{F}\right)$ associe 
\begin{equation}
  \label{eq:IF}
  I_{\fii}(a) : u \in C^{\infty}_c(X) \mapsto \int_{\Gamma}e^{i\fii(x,\theta)}a(x,\theta)u(x)dxd\theta
\end{equation}
est continue pour la topologie $S^m_{\rho,\delta}$ à valeurs dans l'espace des distributions d'ordre strictement plus grand que $\frac{d+m}{\min(\rho,1-\delta)}$.
\end{prop}
\begin{preuve}
La démonstration originale, dans \cite{hormander1971fourier}, utilise des intégrations par parties, mais ici on va pouvoir utiliser directement le lemme de phase non stationnaire, comme dans \cite{hormander2003analysis}.

On commence par une décomposition dyadique de l'espace des phases. On pose $\chi_0$ une fonction lisse sur $\R^d$ qui vaut $1$ lorsque $|\theta|\leq 1$ et qui vaut $0$ lorsque $|\theta| \geq 2$. On définit alors, pour $\nu > 0$ :
\begin{equation*}
  \chi_{\nu}(\theta)=\chi(2^{-\nu}\theta)-\chi(2^{1-\nu}\theta).
\end{equation*}
Les $\chi_{\nu}$ forment alors une partition parafinie de l'unité. Par ailleurs on a, pour une constante $C$ indépendante de $\nu$ :
\begin{equation*}
  |\partial^{\alpha}\chi_{\nu}(\theta)| \leq C(1+|\theta|)^{-|\alpha|}
\end{equation*}
Cette décomposition est donc adaptée à notre problème puisque pour $a \in S^m_{\rho,\delta}$, et pour $K,\alpha,\beta$, il existe une constante $C$ telle que pour tout $\nu$,
\begin{equation*}
  |\partial^{\beta}_x\partial^{\alpha}_{\theta}\chi_{\nu}(\theta)a(x,\theta)| \leq C(1+|\theta|)^{m-\rho|\alpha|+\delta|\beta|}.
\end{equation*}
Et en particulier, si $\gamma = \max(1-\rho,\delta)<1$, on a une constante $C$ indépendante de $\nu$ telle que si $x \in K$ et $|\theta|\in ]1/2,2[$, on a :
\begin{equation*}
  |\partial^{\alpha}_{x\theta}a(x,2^{\nu}\theta)| \leq C2^{\nu(m+\gamma |\alpha|}
\end{equation*}

Or, pour $\nu\geq 0$, on peut écrire, par changement de variables
\begin{equation*}
  I_{\fii}(\chi_{\nu+1}a)(u)=2^{d\nu}\int e^{i2^{\nu}\fii(x,\theta}\chi_1(\theta)a(x,2^{\nu}\theta)u(x)dxd\theta
\end{equation*}

Et on applique alors le lemme de phase non stationnaire pour obtenir, avec $C$ indépendant de $\nu$ :
\begin{equation*}
  |I_{\fii}(\chi_{\nu+1}a)(u)| \leq C2^{\nu(d+m-k(1-\gamma))}\|u\|_{C^k(K)}.
\end{equation*}
Lorsque $k(1-\gamma)>d+m$ et $k\geq 0$, on peut sommer ces estimations et obtenir que $I_{\fii}(a)$ s'étend en une distribution d'ordre $-k$. Lorsque $k<0$, on peut appliquer ce résultat en remplaçant $u$ par $u\fii$ et raisonner par dualité. Par ailleurs les estimations qui apparaissent ne font intervenir que les semi-normes de $a$ dans $S^m_{\rho,\delta}$, d'où le résultat.
\end{preuve}

On peut alors définir des classes d'intégrales :
\begin{defn}
Lorsque $\fii$ est une phase sur $\Gamma$, avec $F\subset \Gamma$ un fermé conique, et $m$ un réel, on note $\In^m_{\fii}(F)$ l'espace des distributions qui peuvent s'écrire sous la forme (\ref{eq:IF}), avec un $a \in S^m(\mathring{F})$. 
\end{defn}

\begin{prop}
	Soit $\fii$ une phase et $a$ un symbole. Alors la distribution $I_{\fii}(a)$ définie par la formule (\ref{eq:IF}) vérifie :
	\begin{equation*}
	\text{sing supp }I_{\fii}(a) \subset \{x \in X,\,\exists \theta,\,(x,\theta)\in F \text{ et } d_{\theta}\fii(x,\theta)=0\}.
	\end{equation*}
\end{prop}
\begin{preuve}
	Posons
	\begin{equation*}
	S=\{x \in X,\,\exists \theta,\,(x,\theta)\in F \text{ et } d_{\theta}\fii(x,\theta)=0\}.
	\end{equation*}
	Pour $x$ en-dehors de $S$, la fonction $\fii$ est une phase en $\theta$ qui dépend du paramètre $x$, pour lequel toutes les expressions à l'intérieur de l'intégrale évoluent de manière lisse. On peut donc répéter la preuve précédente en incorporant un paramètre lisse $x$ et on obtient alors une dépendance lisse par rapport à $x$ de l'intégrale à $x$ fixé.
\end{preuve}
