%%% Local Variables:
%%% TeX-master: ../provisoire
%%% End:
\section{Opérateurs Intégraux de Fourier à phase complexe}
\subsection{Définitions}
Dans cette section, on va développer la théorie des OIF à phase complexe, de manière très similaire aux OIF à phase réelle. Ceci est permis par la construction qu'on a réalisée en passant par les idéaux lagrangiens. De nombreuses constructions données ici ne seront que des reprises des constructions de la section précédente.

On commence par définir une phase positive non dégénérée :
\begin{defn}
	Soient $X$ et $Y$ deux variétés. Une fonction $\fii \in C^{\infty}(X\times Y \times (\R^d\setminus 0))$ est appelée \emph{phase positive non dégénérée} lorsque les conditions suivantes sont satisfaites :
	\begin{itemize}
		\item $\Im \fii \geq 0$;
		\item à $x$ fixé, la fonction $(y,\theta) \mapsto \fii(x,y,\theta)$ n'a pas de point critique ;
		\item à $y$ fixé, la fonction $(x,\theta) \mapsto \fii(x,y,\theta)$ n'a pas de point critique; 
		\item En un point $\gamma$ en $d_{\theta}\fii$ s'annule, les formes $d(\partial \fii/\partial \theta_i)$ sont indépendantes (sur $\C$).
	\end{itemize}
\end{defn}

Une telle phase permet de construire des OIF à phase complexe en tant qu'opérateurs à noyau, comme précédemment.

On constate d'emblée que les considérations de la section précédente sur la variété caractéristique deviennent caduques. En effet, l'ensemble :
\begin{multline*}
C_{\fii}=\{(x,\xi,y,\eta) \in T^*X \times T^*Y,\,\exists \theta\in \R^d\setminus 0,\,\\\dd_{\theta}\fii(x,y,\theta)=0,\,\dd_x\fii(x,y,\theta)=\xi, \dd_y\fii(x,y,\theta)=\eta\}
\end{multline*}
n'est plus une variété lagrangienne parce qu'il y a "trop d'équations". Pour remédier à cela, on peut complexifier localement les variétés, et décrire $C_{\fii}$ comme l'ensemble des points réels d'une variété complexe, comme dans \cite{melin1975fourier}, ou bien on peut continuer à raisonner avec les idéaux lagrangiens. C'est ce que nous allons faire, en suivant à nouveau \cite{hormander1985}.

\subsection{Idéaux lagrangiens positifs}
On peut répéter les arguments précédents en faisant attention à la positivité. On définit premièrement un idéal lagrangien complexe de la manière suivante :

\begin{defn}Soit $S$ une variété symplectique de dimension $2n$. Un idéal $J \subset C^{\infty}(S,\C)$ est dit \emph{lagrangien} lorsque :
	\begin{itemize}
		\item $\{J,J\}\subset J$.
		\item Localement, il existe $n$ générateurs de $J$, dont les différentielles sont $\C$-linéairement indépendantes sur $J$.
		\item Si une fonction $u$ vérifie que pour toute fonction $f \in C^{\infty}_c(S,\C)$, on a $fu \in J$, alors $u \in J$.
	\end{itemize}
	Si les générateurs peuvent être choisis invariants par rapport à une action de $\R^+$ sur $S$, on dit que l'idéal lagrangien est conique.
\end{defn}

Une phase positive non dégénérée $\fii$ engendre un idéal lagrangien $J_{\fii}$ par les constructions identiques au cas réel.

Réciproquement, tous les idéaux lagrangiens coniques complexes sont-ils engendrés par une phase positive ? Pas nécessairement, car on n'a pas encodé la positivité. Clairement, si un idéal $J$ est engendré par une phase positive $\fii$, alors l'idéal $\overline{J}$ des conjuguées des fonctions dans $J$, est par exemple engendré par la phase $\overline{\fii}$, qui n'est pas positive en général.
\begin{defn}
	On dit qu'un idéal lagrangien est \emph{positif} lorsqu'il est engendré par une phase positive.
\end{defn}

On dispose comme dans le cas réel d'un modèle local universel pour ces idéaux :
\begin{lem}
	Soit $Z$ une variété et $J$ un idéal lagrangien conique de $C^{\infty}(T^*Z \setminus 0,\C)$. Soit $m_0\in T^*Z$, alors il existe un voisinage conique $V$ de $m_0$ tel que :
	\begin{enumerate}
		\item Ou bien la restriction à $V$ des fonctions dans $J$ vaut tout l'espace : $J|_V = C^{\infty}(V)$.
		\item Ou bien il existe des coordonnées locales $(x_j,\xi_j)$ sur $V$ et une fonction $H\in C^{\infty}(V)$, homogène de degré 0, qui ne dépend que de $\xi$, et telle que ${\Im H \leq 0}$, telle que $J|_{V}$ est généré par les fonctions
		\begin{equation*}
		x_j-\cfrac{\partial H}{\partial \xi_j}.
		\end{equation*}
	\end{enumerate} 
\end{lem}

On peut alors généraliser les énoncés du théorème de changement de phase et du théorème de composition.

En particulier, remarquons que l'adjoint d'un opérateur associé à la
phase $\fii$ (d'idéal $J$) est un opérateur associé à la phase
$-\overline{\fii}$. En échangeant les places de $x$ et $y$ dans la
définition, $-\overline{\fii}$ engendre un idéal qu'on notera
$J^*$. En particulier, se demander si un OIF est un projecteur orthogonal n'a donc de sens que si $J=J^*=J\circ J$.