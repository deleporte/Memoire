\section{Fibrés positifs sur une variété kählerienne compacte}

\subsection{Sections holomorphes de fibrés positifs}

On va appliquer l'analyse microlocale du projecteur de Szeg\H{o} à la quantification de variétés kähleriennes compactes. On exposera ici les principaux résultats et raisonnements de \cite{Zelditch2000}.

Soit $M$ une variété complexe compacte, et $(L,h)$ un fibré en droites sur $M$ associé à une structure hermitienne. Pour tout $n \in \Z$, $h$ induit une métrique $h_N$ sur $L^{\otimes N}$. Notre but ici est d'étudier les espaces $H^0(M,L^{\otimes N})$, dont on se donne une base orthonormée $(s^N_1, \ldots, s^N_{d_N})$ (on sait que $H^0(M,L^{\otimes N})$ est de dimension finie car $M$ est compacte).

On suppose que $L$ est \emph{strictement positif}, c'est-à-dire que $D=\{(m,v) \in L^{-1}, h_{-1}(v,v) < 1\}$ est fortement pseudoconvexe (cf annexe). Le produit hermitien $h$ définit alors une métrique sur $M$, que l'on va définir maintenant. Soit $V$ un petit ouvert de $M$, $U$ un petit ouvert de $\C^n$, $L_V$ la fibre au-dessus de $V$, et soit une carte locale de la forme
\begin{align*}
	\Psi : U \times \C &\mapsto L_V\\
	(x,z) &\mapsto (\psi(x),za(x))
\end{align*}
où on demande que $\psi$ et $a$ soient holomorphes. De cela on peut déduire une carte locale sur $L^*_V$ :
\begin{align*}
	\Psi^* : U \times \C &\mapsto L^*_V\\
	(x,z) &\mapsto (\psi(x),za^{-1}(x))
\end{align*}
L'équation qui définit $(\Psi^*)^{-1}D$ est alors $|z|^2\leq\|a(x)\|_h$. Comme $X$ est fortement pseudoconvexe, on a nécessairement, sur un voisinage de $(\Psi^*)^{-1}X$ : 
\begin{equation*}
	\Hess_{\C} \left((x,z) \mapsto \log(|z|^2\|a(x)\|_h^{-2}) \right) >>0.
\end{equation*} En effet, il existe une fonction qui définit $(\Psi^*)^{-1}X$ et telle que sa hessienne complexe sur le bord est définie positive. Or, par la proposition \ref{prop:Levi-invariant}, la positivité ne dépend pas de la fonction qui définit le bord.

On en déduit simplement que $-\Hess_{\C} \log \|a(x)\|_h >> 0$. Cette condition est en fait équivalente à la forte pseudoconvexité de $D$, et est parfois considérée comme la définition d'un fibré positif, mais nous avons préféré nous concentrer sur le problème qui nous intéresse.

La forme sesquilinéaire $-\Hess_{\C}\log \|a(x)\|_h$ permet alors de définir une structure riemannienne sur $U$, en prescrivant :
\begin{align*}
	\xi \in T^{1,0}U&\Rightarrow |\xi|=-\Hess_{\C}\log \|a(x)\|_h(\xi,\xi)\\
	\eta \in T^{0,1}U&\Rightarrow |\eta|=-\Hess_{\C}\log \|a(x)\|_h(\overline{\eta},\overline{\eta})\\
	T^{1,0}  U & \perp T^{0,1}U
\end{align*}
En poussant par $\Psi$, on obtient une structure riemannienne sur $V$.

Peut-on recoller les morceaux ? Clairement, si deux cartes $\Psi_1$ et $\Psi_2$, de la forme précédente, sont définies sur le même ouvert, alors sans perte de généralité $\psi_1=\psi_2$. En effet, on peut faire des changements de variable en $x$ seul, et la modification de $\Hess_{\C}\log \|a\|_h$  est alors compatible avec le changement de carte des espaces tangents.

Dans ce cas les fonctions $a_1$ et $a_2$ ne diffèrent que d'un facteur holomorphe $f$, avec sans perte de généralité $f \in \C \setminus \R^-$. Il ne reste plus qu'à voir que :
\begin{align*}
	\Hess_{\C}\log \|fa_1\|_h &= \frac 12 \Hess_{\C} \log f + \frac 12 \Hess_{\C} \log \overline{f} + \Hess_{\C} \log \|a_1\|_h\\
		&= \Hess_{\C} \log \|a_1\|_h
\end{align*}
On peut donc construire une structure riemannienne sur $M$, de manière univoque. 

De manière peut-être encore plus naturelle, la $(1,1)$-forme $-\partial \overline{\partial}\log \|a\|_h$ sur $U$ peut être poussée en une $(1,1)$ forme univoque sur $M$, fermée (car localement exacte), et non dégénérée, autrement dit, une forme symplectique $\omega_g$. Cette forme symplectique est reliée à la structure riemanienne, par $|\xi|=\omega_g(\xi,J\xi)$. La forme symplectique, la structure riemannienne, et la structure complexe sont compatibles, on dit que c'est une structure \emph{kählérienne}.

Le cadre géométrique étant donné, on peut formuler les résultats de l'article \cite{Zelditch2000}. On y démontre l'existence d'un développement asymptotique :

\begin{theorem}[Zelditch] Il existe une famille de fonctions lisses $a_j : M \to \R$ telle que, lorsque $N \to +\infty$, pour tout $R>0$,  
	\begin{equation*}
	\sum_{i=1}^{d_N}\|s^N_i(z)\|^2_{h_N} = N^n + \sum_{0 < j < R} a_j(z)N^{n-j} + O_{C^{\infty}}(N^{n-R})
	\end{equation*}
\end{theorem}

Ce théorème est riche de ses corollaires. Il implique premièrement que pour $N$ assez grand, les sections $s^N_i$ ne s'annulent pas toutes en même temps, on peut alors définir une application projective définie par $\fii_N : M \mapsto \C\mathbb{P}^{d_N-1}$ par $m \mapsto [s^N_1(m) : \ldots : s^N_{d_N}(m)]$, qu'on appelle application de Kodaira. On peut alors démontrer :
\begin{theorem}[Kodaira] (\cite{fritzsche}) Pour $N$ assez grand, $\fii_N$ est un plongement. \end{theorem}

On peut alors comparer la métrique $\omega_g$ avec le pull-back par $\fii_N$ de la métrique de Fubini-Study sur $\C\mathbb{P}^{d_N-1}$. On a en fait :

\begin{theorem}[Tian] \cite{tian1990set}Soit $k \in \N$, alors on a \begin{equation*}
\left\|\cfrac1N \fii_N^*(\omega_{FS}) - \omega_g \right\|_{C^{k}} = O(N^{-1})
\end{equation*}\end{theorem}

Le plongement de Kodaira est donc quasiment isométrique.

\subsection{Espaces de Hardy équivarients}

On notera $X$ la frontière de $D$ ; c'est un $S^1$-fibré sur $M$, et la fonction $\rho(z)=h_{-1}(z,z)-1$ définit $X$. La fonction $\rho$, et l'opérateur $\overline{\partial}_b$, sont alors invariants par l'action de $S^1$. On peut donc décomposer l'espace de Hardy $H^2(X)$ en la somme directe des :
\begin{equation*}
	H^2_N(X)=\{f \in H^2(X), f(r_{\theta}x) = e^{iN\theta}f(x)\}
\end{equation*}

L'espace $H^2_N(X)$ sera appelé \emph{espace de Hardy $N$-équivariant}. On va maintenant relier les fonctions dans cet espace aux sections holomorphes de $L^{\otimes N}$ :

\begin{prop} Soit $N \in \Z$. Alors l'application :

\begin{align*}
	H^0(M,L^{\otimes N}) &\mapsto H^2_N(X) \\
	s &\mapsto \left((m,v) \mapsto \left\langle v^{\otimes N}, s(m)\right\rangle \right)
\end{align*}
est un isomorphisme unitaire.
\end{prop}

Etudier $H^0(M,L^{\otimes N})$ revient donc à étudier les espaces de Hardy $N$-équivariants. On note $S_N$ le projecteur orthogonal de $L^2(X)$ sur $H^2_N(X)$, et encore $S_N(x,y)$ le noyau de ce projecteur, qui n'est rien d'autre que la $N$-ième composante de Fourier de $S(x,y)$ par rapport à sa première variable.

En notant $(s^N_1, \ldots, s^N_{d_N})$ une base orthonormée de $H^0(M,L^{\otimes N})$, et en notant $(\hat{s}^N_1, \ldots, \hat{s}^N_{d_N})$ la base correspondante de $H^2_N(X)$, on a naturellement :
\begin{equation*}
	S_N(x,y)=\sum_{i=1}^{d_N}\hat{s}^N_i(x) \overline{\hat{s}^N_i(y)\strut}
\end{equation*}

En particulier, sur la diagonale,
\begin{equation*}
	S_N(x,x)=\sum_{i=1}^{d_N}\|\hat{s}^N_i(x)\|^2
\end{equation*}

Au point $w$ défini par les coordonnées $[w_0:\ldots:w_m]$, la métrique de Fubini-Study s'écrit :
\begin{equation*}
	\omega_{FS}(w) = \cfrac{i}{2\pi}\,\partial\overline{\partial} \log\left(\sum_{i=0}^m|w_i|^2\right).
\end{equation*}

Ainsi,
\begin{align*}
	\fii_N^*\,\omega_{FS}(m) &= N \omega_g + \cfrac{i}{2\pi}\,\partial\overline{\partial} \log\left(\sum_{j=0}^m\|s^N_i(m)\|_{h_N}^2\right)\\
	&= N\omega_g + \cfrac{i}{2\pi}\,\partial_b\overline{\partial}_b \log S_N(x,x)
\end{align*}

L'avant-dernière ligne mérite une explication. Si $e_L$ est une section locale de L, holomorphe et non nulle, alors avec $a=\|e_L\|_h$, on a $\omega_g = i \partial \overline{\partial}a$ par définition. Par ailleurs on écrit localement $s^N_i=f_i^Ne_L^{\otimes N}$, où les $f^N_i$ sont des fonctions holomorphes au voisinage du point considéré. Alors par définition, 
\begin{equation*}
	\fii_N^*\,\omega_{FS}(m) = \cfrac{i}{2\pi}\,\partial\overline{\partial} \log\left(\sum_{i=0}^m|f^N_i|^2\right).
\end{equation*}
d'où le résultat.

Il ne reste plus qu'à estimer $S_N$ et $\partial_b \overline{\partial}_bS_N$ le long de la diagonale, ce qu'on va faire grâce à la caractérisation ci-dessus.

\subsection{La preuve}
On part de :
\begin{equation*}
	S_N(x,x)=N\int_{R^+_*}\int_{S^1}e^{iN(-\theta + t\psi(r_{\theta}x,x))}\sum_{k=0}^{+\infty}s_k(r_{\theta}x,x)(Nt)^{n-1-k}dtd\theta + O(N^{-\infty})
\end{equation*}
où on a fait le changement de variable $t \mapsto Nt$ dans le but d'appliquer le lemme de phase stationnaire.

On définit alors la phase
\begin{equation*}
	\Phi(x,x,t,\theta)=t\psi(r_{\theta}x,x)-\theta,
\end{equation*}
qui vérifie $Im(\Phi) \geq 0$, Comme $Im(\psi) > 0$ hors de la diagonale, il n'y a pas de point critique sauf si $r_{\theta}x=x$, autrement dit, $\theta=0$.

Le comportement de $\psi$ par rapport à l'action de $S^1$ va être spécifié maintenant. Rappelons que la phase doit vérifier, pour $h\in \C^n$ et $k \in \C^n$ petits, par une carte locale :
\begin{equation}
  \psi(x+h,x+k) =-i\sum_{\alpha,\beta}(\partial^{\alpha}\overline{\partial}^{\beta}\rho)(x)h^{\alpha}\overline{k}^{\beta} + O(h^{\infty},k^{\infty}).
\end{equation}
Pour clarifier l'esprit, donnons-nous une carte locale holomorphe au voisinage $U$ d'une fibre de $L^*$. On a alors une fonction non nulle $a$ telle que $U \cap X \approx \{(m,v)\in \C^{n-1}\times \C, a(m)|v|\leq 1\}$. On a alors $\rho(m,v)=1-a(m)^2|v|^2$. Ainsi, on a $\partial_v \rho(m,v) = a(m^2)\overline{v}$.
Par ailleurs, le champ de vecteurs $\partial_{\theta}$ qui engendre l'action de $S^1$ vaut exactement $\partial_{\theta}(m,v)=(0,iv)$. On a alors que, pour $\theta$ petit, la dépendance en $\theta$ est :
\begin{equation*}
  \psi(r_{\theta} x,x) = \theta + O(\theta^{2}).
\end{equation*}
 Ainsi :
\begin{equation*}
	\Phi(x,x,t,\theta) =(t-1)\theta + O(\theta^2).
\end{equation*}

Les points critiques vérifient $d_t\Psi = d_\theta \Psi =0$, donc $t=1, \theta=0$. En ce point critique, le déterminant de la hessienne vaut -1.

On invoque alors le lemme de phase stationnaire : il existe des opérateurs différentiels $L_{k,j}$, de degré $2k$, indépendants de $N$, tels que, à tout ordre K>0, 

\begin{align*}
	S_N(x,x) &= N^n s_0(x,x) + \sum_{k=1}^{K}\sum_{j=0}^{K-k}N^{n-k-j}L_{k,j}s_j(x,x) + R(N,K,x)\\
		S_N(x,x) &= N^n \cfrac{1}{4\pi^n}\sqrt{\det{L_X,x}}\|d\rho\|^{\frac{n+1}{2}}(x) + \sum_{k=1}^K N^{n-1}a_i(x) + R(N,K,x)
\end{align*}
Ici, on a, pour tous $N$ et $j$,
\begin{equation*}
  \|R(N,K,\cdot)\|_{C^j} \leq N^{n-K-1}C_{j,K}
\end{equation*}
Ceci est l'estimation sur $S_N(x,x)$ désirée. En différentiant cette expression, on obtient $\partial_b\overline{\partial}_b \log(S_N(x,x)) = O(1)$, d'où le corollaire.

\begin{rem}
Comme $S_N$ est un projecteur, on a évidemment :
\begin{equation}
  tr(S_N) = \int_XS_N(x,x)dV=\dim H^2_N(X).
\end{equation}
En particulier, on vient de démontrer l'existence d'un développement asymptotique :
\begin{equation}
  \dim H^2_N(X) = CN^n+\sum_{k=1}^{K}C_jN^{n-j} + O(N^{n-K}).
\end{equation}
Ce résultat est très similaire au théorème de Riemann-Roch. Dans le cas de $\C\mathbb{P}^1$,l'espace des sections de $\mathcal{O}(1)^{\otimes N}$ est en bijection naturelle avec l'espace des fonctions méromorphes avec un pôle d'ordre au plus $N$ en $[0:1]$. Le théorème de Riemann-Roch est en réalité plus puissant car il donne une expression polynômiale de la dimension.
\end{rem}

Remarquons que l'application de Kodaira est non seulement bien définie, mais immersive : si sa différentielle en un point n'était pas injective, le pull-back de la forme de Fubini-Study serait nul à cet endroit.

La démonstration de l'injectivité de l'application de Kodaira est reportée au chapitre suivant ; en effet, elle repose sur l'expression asymptotique de $S_N$ après changement d'échelle.
