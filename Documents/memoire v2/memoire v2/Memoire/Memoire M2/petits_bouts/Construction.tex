\chapter{Structure de Toeplitz}

Cette partie est consacré à un exposé des raisonnements et résultats de \cite{Shiffman2002}. Cependant, l'auteur a préféré, pour des raisons de clarté, exposer les raisonnements sur les Opérateurs Intégraux de Fourier en terme de leurs idéaux, et non de leurs lagrangiennes. Cette différence est motivée par la définition identique des idéaux dans le cas d'une phase complexe, et par la difficulté conceptuelle que pose d'emblée l'introduction des extensions presque analytiques comme présenté dans \cite{melin1975fourier}. On rompt donc avec la description originelle proposée par Boutet et Guillemin.

Dans ce chapitre, on reprend les outils et objets du chapitre précédent, mais on va en présenter une construction alternative, qui n'utilise pas a priori la structure exacte de $\overline{\partial}_b$, telle que la caractérisation microlocale de son noyau, ni même l'existence de solutions. Le but est de pouvoir étendre cette construction, donnée par Boutet et Guillemin dans l'annexe de \cite{BoutetdeMonvel1981}, à des cas où la structure complexe est remplacée par une structure presque complexe non intégrable, dans laquelle $\overline{\partial}_b$ a les mêmes propriétés microlocales, mais ne possède pas de solution globale.

De manière similaire, on se donne $(M,\omega, J)$ une variété symplectique presque complexe de dimension $2n$, et on se donne $(L,h)$ un fibré en droites hermitien sur $M$. Il s'agit maintenant de retrouver dans le cas symplectique la notion de "bonne courbure" pour $L$. Une des manières, décrite dans \cite{woodhouse1997geometric}, est de munir $L$ d'une connexion $\nabla$ telle que les dérivées covariantes de la métrique $h$ s'annule, et dont la courbure vaut $-i\omega$. 

Dans le cas complexe, on peut construire explicitement une telle connexion : si $a$ est une section locale holomorphe non nulle fixée, on considère la connexion qui vérifie $\nabla a = \partial \log \|a\|_h \otimes a$, les dérivées covariantes de la métrique s'annulent ; la courbure de la connexion vaut exactement $-i\omega_g$ selon les constructions du chapitre précédent.

\begin{rem}L'introduction de la connexion $\nabla$ peut paraître artificielle dans ce contexte. En réalité, le cadre naturel d'application de cette théorie est l'étude de la quantification au sens de Kostant dans \cite{kostant1970quantization} et Souriau dans \cite{souriau1967quantification}, où $\nabla$ est alors donné a priori.\end{rem}

A partir de ces données géométriques, on va reprendre le travail du chapitre précédent, mais en fonctionnant à l'envers : après avoir explicité des coordonnées locales, on construit en premier lieu un idéal lagrangien idempotent, puis un projecteur associé, et on définit les espaces de Hardy comme les fonctions équivariantes dans le noyau de ce projecteur.

On notera $g(v,w)=\omega(v,Jw)$ la métrique riemannienne compatible avec $\omega$ et $J$.

\section{Construction du projecteur approximatif}

\subsection{Des bonnes coordonnées locales}

On se donne un point $P_0 \in M$. Au voisinage de ce point, on va considérer des coordonnées locales $(x_1,\ldots, x_n,y_1,\ldots, y_n)$. On écrit $z_j=x_j+iy_j$. Qu'exige-t-on de ces coordonnées ? D'un côté, on aimerait avoir $\partial/\partial z_j \in T^{1,0}M$ pour tout $j$, autrement dit, $\partial/\partial y_j = -J\partial/\partial x_i$ ; de l'autre, on voudrait que ce système de coordonnées envoie $\omega$ sur la forme symplectique canonique $\omega_0 = \sum \dd x_i \wedge \dd y_i$. Malheureusement, ces conditions sont incompatibles sauf si $J$ est intégrable. Néanmoins on peut exiger d'avoir ces conditions en un point.

\begin{defn}
	On dit que ce système de coordonnées est \emph{privilégié} lorsque les deux conditions suivantes sont satisfaites :
	\begin{itemize}
		\item Pour tout $j$, on a $\left.\cfrac{\partial}{\partial z_j}\right|_{P_0} \in T^{1,0}M$.
		\item $\omega(P_0)=\sum \dd x_i \wedge \dd y_i$.
	\end{itemize}
\end{defn}

\begin{rem}
	Sous ces conditions, on a aussi $g(P_0) = \sum \dd x_i \otimes \dd x_i + \dd y_i \otimes \dd y_i$.
\end{rem}

On peut alors corriger les vecteurs pour former une base locale $\overline{Z}_j$ de $T^{0,1}M$, avec :

\begin{equation*}
	\overline{Z}_j = \cfrac{\partial}{\partial \overline{z_j}} + \sum_{k=1}^mB_{j,k}(z)\cfrac{\partial}{\partial z_k},
\end{equation*}
avec $B_{j,k}(P_0)=0$.
Il s'agit maintenant de compléter ce système de coordonnées en un système de coordonnées sur $L$, par une section $e_L$. Quitte à la multiplier par une fonction, on peut supposer que  $\|e_L\|_{P_0}=1$, $\nabla e_L(P_0)=0$, ainsi que $\nabla^2e_L(P_0)=-(g+i\omega)\otimes e_L|_{P_0}$.

Dans les coordonnées locales privilégiées, ceci revient à exiger :
\begin{equation*}
\|e_L\|(z)=1-\cfrac{|z|^2}{2} + O(|z|^3)
\end{equation*}
On notera $a(z)=\|e_L(z)\|^{-2}$, et cette notation sera à nouveau utilisée plus loin.

On peut alors définir une carte locale de $X = \{(m,v)\in L^*, \|v\|_h=1 \}$ par :
\begin{equation*}
(z_1, \ldots, z_n, \theta) \mapsto e^{i\theta}a(z)^{-1/2}e_L^*(z)
\end{equation*}

Les coordonnées locales $(z_1, \ldots, z_n, \theta)$ sur $X$ sont appelées coordonnées de Heisenberg.
 
\subsection{Construction de générateurs}

Notre but est à présent de définir un projecteur de Szeg\H{o} comme un opérateur intégral de Fourier. Celui-ci sera décrit par une phase complexe, il est donc confortable de raisonner en termes d'idéaux lagrangiens. 

L'opérateur de Cauchy-Riemann sur $X$ est un opérateur pseudodifférentiel. Son symbole peut être décrit localement : si $(\overline{Z}_j)$ est la base locale de $T^{0,1}M$ définie précédemment, et $(\zeta_j)$ est la base duale correspondante, le symbole est :
\begin{equation*}
	d_0 := \sigma_{\overline{\partial}_b}(x,\xi) = \sum_{j} \langle \xi, \overline{Z}_j \rangle \zeta_j
\end{equation*}

Notons $p_j(x,\xi)= \langle \xi, \overline{Z_j} \rangle$, et posons $\Sigma = \{(x,\xi) \in T^*X\setminus 0,\,p_j(x,\xi)=0 \forall j\}$ ; c'est un fibré en droites sur $X$. On aimerait définir l'idéal lagrangien en partant de ces fonctions. On rappelle que dans le cas complexe, les fonctions $p_j$ engendrent un idéal $\mathcal{I}$ qui permet de définir l'idéal dont $\db$ est un OIF associé. Pour réaliser cette construction, on a utilisé le fait que l'idéal engendré par ces fonctions était fermé pour le crochet de Poisson, ce qui n'est plus le cas ; en effet :
\begin{equation*}
	\{p_j,p_k\}(\xi) = \langle \xi, [\overline{Z}_j,\overline{Z}_k].
\end{equation*}

Qu'en est-il des conditions de positivité ? Posons $p_{\theta}(\xi) = \langle \xi, \partial/\partial \theta \rangle$, où $\partial/\partial \theta$ dénote la dérivée selon l'action de $S^1$ sur $X$. Alors sur $\Sigma$, la fonction $p_{\theta}$ ne s'annule pas (car si $p_{|theta}=0$ et $p_j=0$, alors $\overline{p_j}=0$, donc $\xi$ est orthogonal à une base de $TX$, donc est nul, absurde.) Par ailleurs, on a choisi nos vecteurs $Z_j$ pour être orthonormés, ainsi, $\frac 1i \{p_j,\overline{p}_k\}=\delta^j_kp_{\theta}$. Cette matrice est définie positive ou bien définie négative selon le signe de $p_\theta$, partitionnant $\Sigma$ en deux fibrés en demi-droite, comme dans le cas complexe. Si on arrive à maintenir cette condition sur nos générateurs modifiés, on pourra continuer notre construction.

Le tenseur de Nijenhuis va nous permettre de construire des générateurs convenables. En effet, on constate que $\{p_j,p_k\}$ est nul sur $\Sigma$. On a donc :
\begin{equation*}
	\{p_j,p_k\} = \sum_{l=1}^nf^l_{jk}p_l + \sum_{l=1}^{n}N^l_{jk}\overline{p_l}
\end{equation*}
On veut corriger les $p_j$ au voisinage de $\Sigma$ pour éliminer le facteur $N^l_{jk}$ au premier ordre. Pour cela, on utilise qu'au voisinage d'un point de $\Sigma$, la fonction $p_{\theta}$ n'est pas nulle. On peut donc poser :
\begin{equation*}
	p_j^{(2)} = p_j + \cfrac{i}{3p_{\theta}}\sum_{k,l}N^l_{jk}\overline{p}_k\overline{p}_l
\end{equation*}
Et alors, comme $\{p_j,\overline{p_k}\}=i\delta^j_k p_{\theta}$, on a que le crochet $\{p_j^{(2)},p_k^{(2)} \}$ est dans l'idéal engendré par les $p_l^{(2)}$ à l'ordre 2 au voisinage de $\Sigma$. On peut réitérer et construire par récurrence le jet sur $\Sigma$ d'une famille de fonctions $p^{\infty}_j$ qui va convenir.

Résumons : on a construit des fonctions $p^{\infty}_j$ qui vérifient que :
\begin{prop}
	L'idéal $\mathcal{I}$ engendré par les fonctions $p^{\infty}_j$ est fermé pour le crochet de Poisson. L'ensemble des zéros communs des fonctions dans $\mathcal{I}$, noté $\Sigma$, se décompose en $\Sigma^+$, sur lequel la matrice $\frac 1i \{p^{\infty}_j,p^{\infty}_k\}$ est définie positive, et $\Sigma^-$, sur lequel cette matrice est définie négative.
\end{prop}

Ces conditions géométriques sont exactement celles qui permettent d'appliquer le raisonnement sur les idéaux lagrangiens évoqué plus haut :

\begin{prop}
	Il existe un unique idéal, noté $\mathcal{J} \in C^{\infty}((T^*X\setminus 0)^2,\C)$, vérifiant toutes les conditions suivantes :
	\begin{itemize}
		\item $\mathcal{I}\otimes 1 + 1 \otimes \overline{\mathcal{I}}\subset\mathcal{J}$;
		\item L'ensemble des points d'annulation de toutes les fonctions dans $\mathcal{J}$ vaut $\diag(\Sigma^+)$;
		\item $\mathcal{J}^*=\mathcal{J}$;
		\item $\mathcal{J}\circ\mathcal{J}=\mathcal{J}$;
	\end{itemize}
\end{prop}

\subsection{Paramétrisation de l'idéal}
Pour montrer que $\mathcal{J}$ est un idéal lagrangien, on va montrer que c'est l'idéal relié à une phase complexe $\psi$, qu'on va construire maintenant.

L'invariance par rapport à $S^1$ et l'antisymétrie de $\mathcal{J}$ nous invitent à considérer une phase sur $X \times X \times \R$ de la forme

\begin{equation*}
	(z, \theta ,w, \phi ,t) \mapsto -it\left(1-e^{i(\theta-\fii)}\cfrac{a(z,w)}{\sqrt{a(z)a(w)}}\right)
\end{equation*}

Où $a(z,w)=\overline{a(w,z)}$.

La raison de la notation $a(z,w)$ vient du fait que $\mathcal{J}$ est contenu dans l'idéal des fonctions qui s'annulent sur $\diag(\Sigma^+)$, ce qui signifie en particulier que $\psi$ s'annule uniquement sur la diagonale, et donc $a(z,z)=a(z)$. La même inclusion donne $d_za$ sur la $\diag(\Sigma^+)$, puisqu'à cet endroit on doit avoir $p^{\infty}_j(z,\theta,d_1\psi)=0$ pour tout $j$. Ces conditions déterminent, de manière unique, le germe de $\psi$ sur la diagonale. 

Peut-on associer à ce germe une phase positive ? On peut vérifier à la main (cf \cite{Shiffman2002}, Lemme 2.4) que :
\begin{equation*}
	a(z,w) = 1 + z\overline{w} + O(|z|^3 + |w|^3).
\end{equation*}
\subsection{Construction d'un projecteur de Szego}

On va construire $S \in I^0(X,X,\mathcal{J})$ vérifiant $S =S^2 = S^*$. On commence naturellement par déterminer le premier terme du symbole : avec $S = \int_{\R^+}e^{it\psi}s_0(x,y)t^{n-1}dt \text{ mod. } I^{-1}(X,X,\mathcal{J})$, on a, grâce au lemme de phase stationnaire :

\begin{equation*}
	s_0(x,y)^2= h(x,y)s_0(x,y) \text{ mod. } \psi
\end{equation*}

Ici $h(x,y)$ est une fonction liée à la hessienne de $\psi$, et non nulle sur la diagonale.

Si on veut que $s_0$ ne s'annule pas, on a un unique candidat naturel modulo $\psi$, à savoir $s_0(x,y)= h(x,y)$.

On peut maintenant appliquer le même raisonnement qu'au chapitre précédent : on dispose d'un projecteur approximatif, lié à un symbole $s_0$, qui vérifie que $s_0 * s_0 = s_0 +r$, avec $r \in S^{-1}$. Si on veut $s=s_0 + e$ avec $s * s = s$, on obtient $e = 1/2 - \sqrt{r + 1/4}$. Cette série formelle possède une réalisation, et l'on peut construire S, qui est unique à un noyau $C^{\infty}$ près.

Quitte à remplacer $S$ par $(S+S^*)/2 \sim S$, on peut supposer que $S$ est autoadjoint. Puisque $S^2=S+K$, et que $K$ est un opérateur compact, on a $spec_{ess}(S)=\{0,1\}$. $S$ possède éventuellement des valeurs propres, qui s'accumulent en $0$ ou en $1$. On peut donc, à un projecteur de dimension finie près, corriger $S$ en un projecteur.

\subsection{Opérateur  pseudo-différentiel associé à S}
On va remplacer $\overline{\partial}_b$ par un opérateur $\overline{D}_0$ tel que $S$ est la projection sur le noyau de $\overline{D}_0$. 

Construisons de ce pas cet opérateur, en suivant la méthode proposée dans \cite{BoutetdeMonvel1981}. On commence par constater que comme le symbole de $\overline{\partial}_b$ est $d_0$, alors $\overline{\partial}_b S$ est de degré au plus $0$. On utilise alors le lemme suivant :

\begin{lem}
	Soit $A \in \mathcal{I}^m(X,X,\mathcal{J})$. Alors il existe un opérateur pseudo-différentiel $Q$ de degré $m$ tel que $A \sim QS$. Si $A$ est symétrique alors on peut choisir $Q$ symétrique.
\end{lem}
\begin{preuve}
	Raisonnons d'abord localement. Au voisinage d'un point $(x,y) \in X\times X$, notons $a$ un symbole de $A$. Si $x \neq y$, alors $A(x,y)$ est $C^{\infty}$ au voisinage de $(x,y)$ donc $Q=0$ convient. Si $x=y$, on sait que $s_0(x,y)$ est non nul au voisinage de $(x,y)$, donc sur ce voisinage on peut écrire $a(x,y,t)=s_0(x,y)q_0(x,y,t)$, où $q_0$ est de degré $m$. Avec $Q_0$ l'opérateur pseudo-différentiel de symbole $q_0$, on a $A-SQ_0 \in \mathcal{I}^{m-1}(X,X,\mathcal{J})$ au voisinage de $(x,y)$. Récursivement, on construit une suite de symboles $q_i$, de degré $m-i$, tel que $A-S(Q_0+Q_1+\ldots + Q_i) \in \mathcal{I}^{m-i-1}(X,X,\mathcal{J})$ au voisinage de $(x,y)$. A la suite formelle $q_i$, on peut associer une réalisation $Q$ par le lemme de Borel. Alors $A-SQ$ est de classe $C^{\infty}$ au voisinage de $(x,y)$. Si $a$ est symétrique alors naturellement la suite $q_i$ est symétrique donc $Q$ est symétrique.
	
	Par une partition de l'unité sur $X \times X$, on peut construire $Q$ globalement.
\end{preuve}

Par application du lemme, on sait alors que $\overline{\partial}_b S \sim QS$, où $Q$ est un opérateur pseudo-différentiel symétrique de degré $0$. Par ailleurs, on peut remplacer $0$ par sa moyenne le long de l'action de $S^1$ (qui lui est équivalente) de manière à ce que $Q$ soit $S^1$-invariant.

On construit alors $D_0 = (\overline{\partial}_b - Q)(Id-S)$. C'est un opérateur symétrique, et pseudodifférentiel puisque $\overline{\partial}_b$ et $Q$ sont pseudodifférentiels et $(\overline{\partial}_b-Q)S \sim 0$; par ailleurs $D_0$ est symétrique, $S^1$-invariant, de symbole $d_0$, et $S$ est la projection sur le noyau de $D_0$.

On peut en fait construire un complexe d'opérateurs qui simulent le complexe $\overline{\partial}_b$ (agissant sur les $(0-q)$-formes), mais nous ne le ferons pas ici.

\section{Asymptotiques dans l'espace de Hardy}

Comme $D_0$ commute avec l'action de $S^1$, on peut décomposer son noyau selon les espaces propres de $\partial_{\theta}$ : pour $N \in \N$, on définit l'espace $H^2_N(X)$ des fonctions dans $L^2(X)$ annulant $D_0$ et $N$-équivariantes. On note par ailleurs  $S_N$ la projection orthogonale sur $H^2_N(X)$, et encore $S_N$ le noyau de ce projecteur, qui est la $N$-ième composante de Fourier de $S(x,y)$ par rapport à sa première variable.

\subsection{Comportement du noyau près de la diagonale}
On dispose du théorème suivant, qui donne le comportement asymptotique du noyau $S_N$ dans un voisinage de la diagonale à l'échelle $\frac{1}{\sqrt{N}}$. A cette échelle, le terme dominant du noyau de Szeg\H{o} est universel, et est donné, dans une bonne carte, par :
\begin{equation*}
	\Pi_H(u,\theta,v,\fii) := \pi^{-n+1}\exp\left(i(\theta-\fii)+i\Im(u\overline{v}) - \frac12 |u-v|^2\right)
\end{equation*}
Plus précisément, on a :
\begin{theorem}[Shiffman-Zelditch]\label{thm:SZ}
	Soit $P_0 \in X$, et $\rho$ une carte locale de coordonnées de Heisenberg autour de $P_0$. On définit $S_N^{P_0}(u,\theta,v,\fii) = S_N(\rho(u,\theta), \rho(v,\fii))$. Soit $K>0$, alors on a :
	\begin{multline*}
	N^{-n+1}S_N^{P_0}\left(\cfrac{u}{\sqrt{N}},\cfrac{\theta}{N}, \cfrac{v}{\sqrt{N}}, \cfrac{\fii}{N}\right) = \\\Pi_H(u,\theta,v,\fii)\left(1+\sum_{r=1}^K N^{-r/2}b_r(u,v) + N^{-(K+1)/2}R_K(u,v,N)\right)
	\end{multline*}
	
	Où $\|R_K(u,v,N)\|_{C^j(\{|u| < r, |v| < r\})} \leq C_{K,j,r}$, indépendamment de $P_0$.
\end{theorem}

\begin{preuve}
	On expose schématiquement la preuve proposée dans \cite{Shiffman2002}. Pour des raisons de lourdeur, les estimations explicites sur les restes ne seront pas reprises.
	
	Par $S^1$-équivariance, il suffit de considérer les cas où $\theta = \fii = 0$. La notation $\theta$ sera à présent utilisée pour une variable muette d'intégration sur la fibre. Le théorème de Boutet et Sj\"ostrand nous donne, pour un certain symbole $s^{P_0}$, et après changement de variable :
	
	\begin{equation*}
	S_N^{P_0}\left(\cfrac{u}{\sqrt{N}},0,\cfrac{v}{\sqrt{N}},0\right) = N \iint e^{iN(-\theta + t\psi\left(\frac{u}{\sqrt{N}},\theta,\frac{v}{\sqrt{N}},0\right))}s^{P_0}\left(\cfrac{u}{\sqrt{N}},\theta, \cfrac{v}{\sqrt{N}},0,Nt\right)d\theta dt
	\end{equation*}
	Le symbole $s^{P_0}$ est classique et admet un développement asymptotique jusqu'à tout ordre $K$ fixé :
	\begin{align*}
		s^{P_0}(\cfrac{u}{\sqrt{N}},\theta,\cfrac{v}{\sqrt{n}},0,Nt)&=N^{n-1}\cfrac{\|d\rho\|^{\frac{n+1}{2}}(P_0)\sqrt{\det{L_X}}}{4\pi^n}\\
		&+ N^{n-1}\sum_{k=0}^{K}N^{-k}s^{P_0}_k(u,\theta,v,\fii)\\
		&+O(N^{n-K'-2})
	\end{align*}
	Et on va appliquer le lemme de phase stationnaire à chaque terme. Commençons par le premier terme, d'ordre $N^{n-1}$. On rappelle que 
	\begin{equation*}
	\psi(u,\theta,v,0) = i\left(1-\cfrac{a(u,v)}{\sqrt{a(u)}\sqrt{a(v)}}\,e^{i\theta}\right),
	\end{equation*}
	et qu'à ce titre, en vertu du développement asymptotique de $a$ au voisinage de la diagonale, la phase dans l'intégrale vaut :
	\begin{align*}
	\widetilde{\Psi} &:= t\psi\left(\cfrac{u}{\sqrt{N}},\theta,\cfrac{v}{\sqrt{N}},0\right) -\theta\\&= it(1-e^{i\theta}) - \theta - \cfrac{it}{N}e^{i\theta}\left(u\cdot\overline{v} - \cfrac {|u|^2+|v|^2}{2}\right) + t e^{i\theta}O(N^{-3/2})
	\end{align*}
	
	On note $\Psi(t,\theta) = it(1-e^{i\theta})-\theta$. La phase dans l'intégrale définissant $S_N^{P_0}$ se décompose donc en une phase oscillante $e^{iN\Psi}$, à laquelle on va appliquer le lemme de phase stationnaire, et une phase non oscillante qu'on va incorporer dans le symbole. Pour se débarasser des ennuis quand $t \to +\infty$, on va également décomposer l'intégrale en une intégrale sur $[0,3]$ et une intégrale sur $[2,+\infty[$, avec une partition de l'unité $\rho_1,\rho_2$ adaptée. Posons :
	\begin{equation*}
		A(t,\theta, u,v,N) = \rho_1(t)e^{iN(\tilde{\Psi}-\Psi)},
	\end{equation*} 
	Et appliquons le lemme de phase stationnaire à l'intégrale :
	\begin{equation*}
		N\int_0^3\int_0^{2\pi}e^{iN\Psi(t,\theta)}A(t,\theta, u,v,N)d\theta  dt
	\end{equation*}
	(qui est l'intégrale du terme dominant, à une constante $\frac{1}{\pi^{n-1}}$ près).
	
	Le seul point critique de $\Psi$ est en $t=1, \theta=0$, et la hessienne en ce point est $\begin{pmatrix}
	0&1\\1&i
	\end{pmatrix}$, de déterminant $-1$. Le lemme de phase stationnaire donne :
	\begin{equation*}
	I_1(u,v,N) =\sqrt{-2i\pi}\sum_{r=0}^K N^{-r}L_r(A)(1,0,u,v,N) + R_K(u,v,N)
	\end{equation*}
	Ici $L_r$ est un opérateur d'ordre $2r$ qui n'agit que sur les variables $t$ et $\theta$, et $L_0=Id$. 
	
	Ceci nous donne bien le développement asymptotique souhaité, il ne reste plus qu'à vérifier que l'autre morceau, $S_N^{P_0}$, est d'ordre $N^{-\infty}$. Mais sur $[2,+\infty[$, la différentielle de $\Psi$ est uniformément loin de $0$ quand $N$ est grand, ce qui permet de conclure par le lemme de phase non stationnaire.
	
	Le premier terme est donc $\sqrt{2i\pi}A(1,0,u,v,\infty) = 1$.
	
	On peut appliquer le même raisonnement aux autres termes du développement de $s^{P_0}$, qui ne vont créer que des termes d'ordre plus petit.
	\end{preuve}
	
\subsection{Applications à Kodaira et Tian}
	Le théorème précédent nous redonne directement l'asymptotique de $S_N$ sur la diagonale qui permet de démontrer que l'application de Kodaira existe, et qu'elle est quasi-isométrique (c'est donc a fortiori un plongement). Pour démontrer que cette application est injective, Shiffman et Zelditch citent une idée de Bouche, que nous réexposons ici. On sait qu'en particulier, pour tout point $P \in X$, la fonction $x \mapsto S_N(P,x)$ est dans $H^2_N(X)$ et de norme 1. Par ailleurs, si $P_N$ et $Q_N$ ont la même image par le plongement de Kodaira, alors en particulier $S_N(P_N,x)=S_N(Q_N,x)$ pour tout $x$. Supposons qu'un tel couple $(P_N,Q_N)$ existe pour une infinité de valeurs de $N$, quitte à extraire on peut raisonner par disjonction de cas : notons $d$ la distance riemanienne ou k\"ahlerienne sur $X$ :
	
	\begin{itemize}
		\item Si $\sqrt{N}d(P_N,Q_N) \to +\infty$, alors par le théorème de la section précédente, on sait que $N^{1-n}\int_{B(P_N,dist(P_N,Q_N)/2)}|S_N^{P_N}|^2 > 1-o(1)$, de même pour $Q_N$. Dans ce cas, la norme $L^2$ de $S_N(x,\cdot)$ est au moins $2N^{(1-n)/2}$, ce qui contredit l'asymptotique du théorème de Zelditch.
		\item Si $\sqrt{N}d(P_N,Q_N) \leq C$, alors l'expression asymptotique de $S_N$ autour de la diagonale, à l'échelle $1/\sqrt{N}$ donne $S_N(P_N,Q_N)\neq N^{1-n}=S_N(Q_N,Q_N)$, et a fortiori, il est interdit que les fonctions $S_N(P_N,\cdot)$ et $S_N(Q_N,\cdot)$ soient identiques
	\end{itemize}

Ce qui conclut la preuve de l'injectivité, on vient donc de démontrer le théorème de Kodaira.

On dispose donc d'un plongement symplectique de variétés à classe de cohomologie entière, dans une variété bien connue. Nous aimerions citer une remarque de S. Kharmalov, par analogie avec les pinceaux de Lefschetz : en pullbackant l'intersection de l'image de la variété avec un sous-espace projectif de dimension paire, on obtient, sous condition de transversalité, des familles de sous-variétés symplectiques de dimension quelconque. L'intérêt est qu'il est en général très difficile de construire des sous-variétés symplectiques d'une variété donnée. 