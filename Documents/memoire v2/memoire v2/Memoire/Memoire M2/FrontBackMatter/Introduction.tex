\chapter*{Introduction}
Durant les dernières décennies, un nouveau parallèle a émergé entre la géométrie différentielle et l'analyse spectrale. Cette association a été motivée par l'intuition physique d'une correspondance entre des phénomènes relevant de la mécanique classique (géodésiques, ou plus généralement flots hamiltoniens) et l'asymptotique de phénomènes quantiques (lorsque $h \to 0$ ou lorsque l'énergie du système devient très grande). Nos recherches ont pour origine un article de B. Douçot et P. Simon, \cite{douccot1998semiclassical}, sur le fond du spectre dans la limite semiclassique de réseaux de spins ferromagnétiques ou antiferromagnétiques. Le cadre originel dans lequel les outils semiclassiques ont été développés (\cite{helffer1988semi},  \cite{zworski2012semiclassical}) est celui de la quantification de Weyl sur $\R^{2n}$, et dans une moindre mesure des opérateurs de type $-h^2\Delta + V$ sur des variétés riemanniennes. Cependant, le cadre théorique de la quantification des spins (\cite{ma2012berezin}, \cite{borthwick2000introduction}) est différent. Il existe à ce jour trop peu de résultats similaires au cas de la quantification de Weyl (on peut citer \cite{borthwick1998semiclassical}), s'agissant en particulier de l'étude des états fondamentaux. Douçot et Simon conjecturent cependant une généralisation des résultats de \cite{helffer1988semi}. 

Dans ce mémoire, nous avons cherché à étudier l'asymptotique du noyau de Szeg\H{o}, dont nous pensons qu'il sera utile pour démontrer des résultats en ce sens. Un tel développement asymptotique a été proposé par \cite{Shiffman2002}, où il sert d'autres fins. Ce mémoire a été pensé comme une introduction bibliographique au futur travail doctoral. En ce sens, plutôt que d'utiliser cette formule pour transposer les résultats de \cite{helffer1988semi}, nous nous sommes donné pour but de comprendre les tenants et aboutissants de ce développement asymptotique et de ses conséquences directes. Il ne sera donc pas directement question ici de quantification.

Dane les années 1970, L. Boutet de Monvel et J. Sj\"ostrand (\cite{BoutetdeMonvel1975}) ont établi un théorème donnant la structure analytique du projecteur sur le noyau d'un certain opérateur différentiel apparaissant en géométrie complexe, appelé projecteur de Szeg\H{o}. Ces travaux ont fait suite à un important travail d'étude de la géométrie complexe par le biais des opérateurs différentiels associés, et notamment les résultats de J.J. Kohn (\cite{kohn1965extension}). Le contexte théorique de \cite{BoutetdeMonvel1975} est celui des Opérateurs Intégraux de Fourier, généralisations des opérateurs pseudodifférentiels. Il nous a semblé judicieux de consacrer un chapitre entier à l'introduction de ces outils, en suivant le formalisme de H\"ormander (\cite{hormander2007}, \cite{hormander1985},\cite{hormander2003analysis}) puisque c'est celui dont se servent Boutet de Monvel et Sj\"ostrand. Il s'agit d'abord de définir des intégrales à phase oscillante comme des distributions, puis d'étudier le comportement des opérateurs dont le noyau-distribution est une telle intégrale. Nous raisonnons par degrés de généralisation successifs, en étudiant d'abord les opérateurs pseudodifférentiels, puis les OIF à phase réelle, et enfin les OIF à phase complexe.

Les prérequis de géométrie complexe ont, eux, été placés en annexe.

Dans le chapitre 2, on démontre le théorème de Boutet et Sj\"ostrand, avant d'en donner une application, qui a fait l'objet d'un article de S. Zelditch (\cite{Zelditch2000}). En effet, d'un développement du projecteur de Szeg\H{o} en somme d'opérateurs de degré décroissant, on obtient un développement asymptotique par rapport à un petit paramètre en codiagonalisant avec un autre opérateur, dont le spectre est $\Z$. On a en particulier un développement asymptotique de la trace, donc de la dimension de l'espace propre commun ; l'interprétation géométrique de cet espace propre conduit à une forme faible du théorème de Riemann-Roch. L'estimation de la diagonale du projecteur fournit par ailleurs une preuve purement analytique de l'existence d'une application immersive, et presque isométrique d'une variété k\"ahlerienne ayant de bonnes propriétés dans l'espace projectif $\C\mathbb{P}^N$ pour $N$ assez grand.

Dans le chapitre 3, on expose deux développements de cette idée, comprises dans un article de B. Shiffman et S. Zelditch (\cite{Shiffman2002}). Le premier est un développement plus précis du noyau de Szeg\H{o}, qui permet notamment de montrer l'injectivité de l'application construite plus haut. C'est ce développement qui nous intéresse dans le cadre de la quantification. Le second développement concerne une construction "à l'envers" : on peut recopier la structure du projecteur de Szeg\H{o} pour créer de toutes pièces un projecteur sur des variétés qui ne sont plus complexes, mais seulement symplectiques (ce qui est d'une grande pertinence si notre but est d'étudier la correspondance classique quantique). Une motivation supplémentaire est donnée par le fait que la seule condition géométrique nécessaire à la construction de ce projecteur "artificiel" est la condition nécessaire à la quantification géométrique au sens de Kostant et Souriau (\cite{Woodhouse1997geometric}).

\chantier
