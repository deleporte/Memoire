\chapter*{Introduction}
Durant les dernières décennies, un nouveau parallèle a émergé entre la géométrie différentielle et l'analyse spectrale. Cette association a été motivée par l'intuition physique d'une correspondance entre des phénomènes relevant de la mécanique classique (géodésiques, ou plus généralement flots hamiltoniens) et l'asymptotique de phénomènes quantiques (lorsque $h \to 0$ ou lorsque l'énergie du système devient très grande). Nos recherches ont pour origine un article de B. Douçot et P. Simon, \cite{douccot1998semiclassical}, sur le fond du spectre dans la limite semiclassique de réseaux de spins ferromagnétiques ou antiferromagnétiques. Le cadre originel dans lequel les outils semiclassiques ont été développés (\cite{hellfer1988semi}, \cite{brummelhuis1991}, \cite{zworski2012semiclassical}) est celui de la quantification de Weyl sur $\R^{2n}$, et dans une moindre mesure des opérateurs de type $-h^2\Delta + V$ sur des variétés riemanniennes. Cependant, le cadre théorique de la quantification des spins (\cite{ma2012berezin}, \cite{borthwick2000introduction}) est différent. Il existe à ce jour trop peu de résultats similaires au cas de la quantification de Weyl (\cite{borthwick1998semiclassical}), s'agissant en particulier de l'étude des états fondamentaux. Douçot et Simon conjecturent cependant une généralisation des résultats de Hellfer et Sjostrand dans \cite{helffer1988semi}. 

Dans ce mémoire, nous avons cherché à étudier l'asymptotique du noyau de Szeg\H{o}, dont nous pensons qu'il sera utile pour démontrer des résultats en ce sens. Un tel développement asymptotique a été proposé par \cite{Shiffman2002}, ou il sert d'autres fins. Ce mémoire a été pensé comme une introduction bibliographique au futur travail doctoral. En ce sens, plutôt que d'utiliser cette formule pour transposer les résultats de \cite{helffer1988semi}, nous nous sommes donné pour but de comprendre les tenants et aboutissants de ce développement asymptotique et de ses conséquences directes. Il ne sera donc pas directement question ici de quantification