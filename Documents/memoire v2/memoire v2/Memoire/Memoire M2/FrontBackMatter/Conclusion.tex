\chapter*{Conclusion}
Les applications du théorème de Kodaira en géométrie complexe sont nombreuses. Un théorème de Chow assure en particulier que toute sous-variété complexe de $\C\mathbb{P}^N$ est une variété algébrique, permettant une classification très rigide des variétés k\"ahlerniennes à classe de cohomologie entière. La validité du théorème de Kodaira, \emph{mutatis mutandis}, dans le cadre symplectique, pourrait permettre d'aider à la classification des variétés symplectiques.

La forme précisée du noyau de Szeg\H{o}, qui fait l'objet du théorème \ref{thm:SZ}, permet par ailleurs de définir et d'étudier une forme de quantification, dite \emph{quantification de Toeplitz}. Sur une variété munie d'une structure de Toeplitz, et en particulier de la suite de projecteurs $S_N$, on peut associer à une fonction lisse $h$ l'opérateur $H_N=S_NM_hS_N$, où $M_h$ est l'opérateur de multiplication par le pullback de $h$ sur le fibré en cercles. En particulier, si la variété est la sphère $\C\mathbb{P}^1$, alors ce procédé est cohérent avec les modèles de spins.

Une étude récente (\cite{le2014theorie}) permet de relier, dans le cas d'une courbe complexe, le spectre de $H_N$ aux propriétés géométriques de $h$ (points critiques et longueur des orbites périodiques du flot). Les résultats de \cite{borthwick1998semiclassical} permettent, eux, de mieux comprendre le spectre de $H_N$ aux énergies qui sont des valeurs régulières de $h$. 

Il semble naturel d'essayer de compléter ces résultats par une étude aux énergies proches de $\min(h)$. Un calcul formel semble conforter une conjecture faite par analogie avec \cite{helffer1988semi}, au moins dans le cas le plus simple :
\begin{conj}
	Soit $h\geq 0$ admettant un seul point d'annulation, tel que la hessienne de $h$ en ce point soit définie positive. Alors pour $N$ assez grand, $s_N=\min \Sp H_N$ est une valeur propre simple ; il existe un réel strictement positif $\lambda_0$ qui ne dépend que de la hessienne de $h$ en ce point, et une suite $(\lambda_i)$ de réels, tels que pour tout $K$, $s_N=N^{-1}(\lambda_0+\sum_{k=1}^K N^{-i}\lambda_i) + O(N^{-K-2})$. Par ailleurs il existe une constante $C$ telle que $s_N$ est le seul élément de $\Sp H_N$ dans $]N^{-1}\lambda_0-CN^{-3/2},N^{-1}\lambda_0+CN^{-3/2}[$.
	
	Par ailleurs, la suite de fonctions propres normalisées associées à $s_N$ se concentre uniquement en l'unique point d'annulation de $h$.
\end{conj}

Lorsque $h$ s'annule en un nombre fini d'endroits, à chaque fois de
manière non dégénérée, on peut établir une conjecture similaire. En particulier, la suite des fonctions propres normalisées associées à l'état fondamental ne semble se concentrer qu'en les points qui minimisent $\lambda_0$, donc dont la hessienne est "plus petite", un phénomène de sélection attesté dans le cas des opérateurs de Schr\"odinger.

On peut alors légitimement se demander s'il existe des phénomènes de type effet tunnel (dans le cas d'un double puits, les deux premières valeurs propres pourraient être exponentiellement proches), ou encore des effets de seconde quantification, si l'ensemble sur lequel $h$ est nul est une sous-variété.