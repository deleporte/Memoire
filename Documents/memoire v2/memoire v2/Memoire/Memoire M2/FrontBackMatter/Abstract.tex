% Abstract

\pdfbookmark[1]{Abstract}{Abstract} % Bookmark name visible in a PDF viewer

\begingroup
\let\clearpage\relax
\let\cleardoublepage\relax
\let\cleardoublepage\relax

\chapter*{Résumé}
	Les fonctions holomorphes dans un ouvert pseudoconvexe d'une variété complexe, lisses jusqu'au bord, voient leur trace vérifier une certaine EDP, appelée équation de Cauchy-Riemann ; le projecteur de Szeg\H{o} est alors  défini comme le projecteur orthogonal, dans l'espace des fonctions $L^2$ sur
	le bord, sur le noyau de cette EDP (appelé espace de Hardy).
	
	Boutet de Monvel et Sjöstrand ont établi que ce projecteur possède une structure analytique qui s'exprime selon un Opérateur Intégral de Fourier, avec une phase bien déterminée. Ceci permet de définir et d'étudier les opérateurs de Toeplitz, qui sont la restriction d'opérateurs pseudo-différentiels à l'espace de Hardy. Sous condition d'invariance par rapport à une action de $S^1$, la structure de Toeplitz donne un moyen naturel de quantification sur la variété quotient ; la caractérisation analytique permet alors une étude quantitative du spectre du hamiltonien quantique.
\endgroup			

\vfill