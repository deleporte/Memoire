% Abstract

\pdfbookmark[1]{Abstract}{Abstract} % Bookmark name visible in a PDF viewer

\begingroup
\let\clearpage\relax
\let\cleardoublepage\relax
\let\cleardoublepage\relax

\chapter*{Résumé}
	Les fonctions holomorphes dans un ouvert pseudoconvexe d'une
        variété complexe, lisses jusqu'au bord, voient leur valeur au bord vérifier une certaine EDP, appelée équation de Cauchy-Riemann ; le projecteur de Szeg\H{o} est alors  défini comme le projecteur orthogonal, dans l'espace des fonctions $L^2$ sur
	le bord, sur le noyau de cette EDP (appelé espace de Hardy).
	
	Boutet de Monvel et Sjöstrand ont établi que ce projecteur
        s'exprime selon un Opérateur Intégral de Fourier à phase complexe \cite{BoutetdeMonvel1975}. L'étude de cette structure analytique permet de retrouver deux théorèmes de la géométrie algébrique complexe : le théorème du plongement de Kodaira, et le théorème de quasi-isométrie de Tian.
	
	Nous établissons l'outil technique principal pour cette étude,
        les Opérateurs Intégraux de Fourier. Pour cela, nous suivons
        les constructions de H\"ormander \cite{hormander2003analysis,
          hormander2007, hormander1985}. En particulier, nous traitons les OIF par leur idéal lagrangien. Nous reprenons les constructions de l'article \cite{BoutetdeMonvel1975} pour introduire les raisonnements de l'article \cite{Zelditch2000}, qui montre un développement asymptotique de la diagonale du noyau de Szeg\H{o} équivariant, et démontre ainsi l'existence de l'application de Kodaira et sa quasi-isométrie. Nous suivons enfin les raisonnements de l'article \cite{Shiffman2002} afin d'établir une construction similaire dans le cas où la variété est symplectique. Une estimation plus précise sur le noyau de Szeg\H{o} est démontrée, qui permet de démontrer l'injectivité de l'application de Kodaira.
\endgroup			

\vfill
