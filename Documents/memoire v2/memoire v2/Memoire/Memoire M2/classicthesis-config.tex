%%%%%%%%%%%%%%%%%%%%%%%%%%%%%%%%%%%%%%%%%
% Thesis Configuration File
%
% The main lines to change in this file are in the DOCUMENT VARIABLES
% section, the rest of the file is for advanced configuration.
%
%%%%%%%%%%%%%%%%%%%%%%%%%%%%%%%%%%%%%%%%%

%----------------------------------------------------------------------------------------
%	DOCUMENT VARIABLES
%	Fill in the lines below to enter your information into the thesis template
%	Each of the commands can be cited anywhere in the thesis
%----------------------------------------------------------------------------------------

% Add drafting for the '[ Date - classicthesis version 4.0 ]' text at the bottom of every page
\PassOptionsToPackage{eulerchapternumbers,listings, pdfspacing,dottedtoc, subfig,parts}{classicthesis}

\usepackage{etoolbox}
% Available options: drafting parts nochapters linedheaders eulerchapternumbers beramono eulermath pdfspacing minionprospacing tocaligned dottedtoc manychapters listings floatperchapter subfig
% Adding 'dottedtoc' will make page numbers in the table of contents flushed right with dots leading to them

\newcommand{\myTitle}{Projecteur de Szeg\"o et th\'eor\`eme de Kodaira\xspace}
\newcommand{\mySubtitle}{M\'emoire de Master\xspace}
\newcommand{\myDegree}{M. \xspace}
\newcommand{\myName}{Alix Deleporte\xspace}
\newcommand{\myProf}{Nalini Anantharaman\xspace}
%\newcommand{\myOtherProf}{Put name here\xspace}
%\newcommand{\mySupervisor}{Nalini Anantharaman\xspace}
\newcommand{\myFaculty}{Put data here\xspace}
\newcommand{\myDepartment}{LMO\xspace}
\newcommand{\myUni}{Université Paris-Sud\xspace}
\newcommand{\myLocation}{Universit\'e de Strasbourg\xspace}
\newcommand{\myTime}{Septembre 2015\xspace}
\newcommand{\myVersion}{version alpha\xspace}

%----------------------------------------------------------------------------------------
%	USEFUL COMMANDS
%----------------------------------------------------------------------------------------

%------------------------
%  CHOOSE MODE
%------------------------
\newbool{print}
%uncomment the line below to make a printed version of the document
%\booltrue{print}

\ifbool{print}
	{\PassOptionsToPackage{draft}{hyperref}
		\PassOptionsToPackage{light}{kpfonts}
	}{}

\newcommand{\cad}{c.-\,à.-\,d}

\newcounter{dummy} % Necessary for correct hyperlinks (to index, bib, etc.)
\providecommand{\mLyX}{L\kern-.1667em\lower.25em\hbox{Y}\kern-.125emX\@}

%----------------------------------------------------------------------------------------
%	PACKAGES
%----------------------------------------------------------------------------------------

\usepackage{lipsum} % Used for inserting dummy 'Lorem ipsum' text into the template for beta testing

%------------------------------------------------
 
\PassOptionsToPackage{utf8}{inputenc} % latin9 (ISO-8859-9) = latin1+"Euro sign"
\usepackage{inputenc}
 
 %------------------------------------------------

\PassOptionsToPackage{frenchb}{babel}  % Change this to your language(s)
\usepackage{babel}

%------------------------------------------------			

\PassOptionsToPackage{square,numbers}{natbib}
\usepackage{natbib}
 
 %------------------------------------------------

%\PassOptionsToPackage{fleqn}{amsmath} % Left-align equations (but why should you do that ?)
 % using this package will automatically configure things from thmconfig.tex
%\usepackage{amsmath}
 
 %------------------------------------------------

\PassOptionsToPackage{T1}{fontenc} % T2A for cyrillics
\usepackage{fontenc}

%------------------------------------------------

\usepackage{xspace} % To get the spacing after macros right

%------------------------------------------------

\usepackage{mparhack} % To get marginpar right

%------------------------------------------------

\usepackage{fixltx2e} % Fixes some LaTeX stuff 

%------------------------------------------------

\PassOptionsToPackage{smaller}{acronym} % Include printonlyused in the first bracket to only show acronyms used in the text
\usepackage{acronym} % nice macros for handling all acronyms in the thesis

%------------------------------------------------

\renewcommand*{\acsfont}[1]{\textssc{#1}} % For MinionPro
%\renewcommand{\bflabel}[1]{{#1}\hfill} % Fix the list of acronyms
%% The \bflabel command is not any more implemented (2015)

%------------------------------------------------

\PassOptionsToPackage{pdftex}{graphicx}
\usepackage{graphicx} 

%------------------------------------------------

\usepackage{todonotes}

%----------------------------------------------------------------------------------------
%	FLOATS: TABLES, FIGURES AND CAPTIONS SETUP
%----------------------------------------------------------------------------------------

\usepackage{tabularx} % Better tables
\setlength{\extrarowheight}{3pt} % Increase table row height
\newcommand{\tableheadline}[1]{\multicolumn{1}{c}{\spacedlowsmallcaps{#1}}}
\newcommand{\myfloatalign}{\centering} % To be used with each float for alignment
\usepackage{caption}
\captionsetup{format=hang,font=small}
\usepackage{subfig}  

%----------------------------------------------------------------------------------------
%	CODE LISTINGS SETUP
%----------------------------------------------------------------------------------------

\usepackage{listings} 
%\lstset{emph={trueIndex,root},emphstyle=\color{BlueViolet}}%\underbar} % for special keywords
\lstset{language=[LaTeX]Tex, % Specify the language for listings here
keywordstyle=\color{RoyalBlue}, % Add \bfseries for bold
basicstyle=\small\ttfamily, % Makes listings a smaller font size and a different font
%identifierstyle=\color{NavyBlue}, % Color of text inside brackets
commentstyle=\color{Green}\ttfamily, % Color of comments
stringstyle=\rmfamily, % Font type to use for strings
numbers=left, % Change left to none to remove line numbers
numberstyle=\scriptsize, % Font size of the line numbers
stepnumber=5, % Increment of line numbers
numbersep=8pt, % Distance of line numbers from code listing
showstringspaces=false, % Sets whether spaces in strings should appear underlined
breaklines=true, % Force the code to stay in the confines of the listing box
%frameround=ftff, % Uncomment for rounded frame
frame=single, % Frame border - none/leftline/topline/bottomline/lines/single/shadowbox/L
belowcaptionskip=.75\baselineskip % Space after the "Listing #: Desciption" text and the listing box
}

%----------------------------------------------------------------------------------------
%	HYPERREFERENCES
%----------------------------------------------------------------------------------------

\PassOptionsToPackage{pdftex,hyperfootnotes=false,pdfpagelabels,bookmarks}{hyperref}
\usepackage{hyperref}  % backref linktocpage pagebackref
\pdfcompresslevel=9
\pdfadjustspacing=1

\hypersetup{
% Uncomment the line below to remove all links (to references, figures, tables, etc)
%draft, 
colorlinks=true, linktocpage=true, pdfstartpage=3, pdfstartview=FitV,
% Uncomment the line below if you want to have black links (e.g. for printing black and white)
%colorlinks=false, linktocpage=false, pdfborder={0 0 0}, pdfstartpage=3, pdfstartview=FitV, 
breaklinks=true, pdfpagemode=UseNone, pageanchor=true, pdfpagemode=UseOutlines,
plainpages=false, bookmarksnumbered, bookmarksopen=true, bookmarksopenlevel=1,
hypertexnames=true, pdfhighlight=/O, urlcolor=webbrown, linkcolor=RoyalBlue, citecolor=webgreen,
%------------------------------------------------
% PDF file meta-information
pdftitle={\myTitle},
pdfauthor={\textcopyright\ \myName, \myUni, \myFaculty},
pdfsubject={},
pdfkeywords={},
pdfcreator={pdfLaTeX},
pdfproducer={LaTeX with hyperref and classicthesis}
%------------------------------------------------
}   

%----------------------------------------------------------------------------------------
%	BACKREFERENCES
%----------------------------------------------------------------------------------------

\usepackage{ifthen} % Allows the user of the \ifthenelse command
\newboolean{enable-backrefs} % Variable to enable backrefs in the bibliography
\setboolean{enable-backrefs}{false} % Variable value: true or false

\newcommand{\backrefnotcitedstring}{\relax} % (Not cited.)
\newcommand{\backrefcitedsinglestring}[1]{(Cited on page~#1.)}
\newcommand{\backrefcitedmultistring}[1]{(Cited on pages~#1.)}
\ifthenelse{\boolean{enable-backrefs}} % If backrefs were enabled
{
\PassOptionsToPackage{hyperpageref}{backref}
\usepackage{backref} % to be loaded after hyperref package 
\renewcommand{\backreftwosep}{ and~} % separate 2 pages
\renewcommand{\backreflastsep}{, and~} % separate last of longer list
\renewcommand*{\backref}[1]{}  % disable standard
\renewcommand*{\backrefalt}[4]{% detailed backref
\ifcase #1 
\backrefnotcitedstring
\or
\backrefcitedsinglestring{#2}
\else
\backrefcitedmultistring{#2}
\fi}
}{\relax} 

%----------------------------------------------------------------------------------------
%	AUTOREFERENCES SETUP
%	Redefines how references in text are prefaced for different 
%	languages (e.g. "Section 1.2" or "section 1.2")
%----------------------------------------------------------------------------------------

\makeatletter
\@ifpackageloaded{babel}
{
\addto\extrasfrenchb{
\renewcommand*{\paragraphautorefname}{Paragraphe}
\renewcommand*{\subparagraphautorefname}{Sous-paragraphe}
\renewcommand*{\footnoteautorefname}{Note de bas de page}
\renewcommand*{\FancyVerbLineautorefname}{Ligne}
\renewcommand*{\theoremautorefname}{Théorème}
\renewcommand*{\appendixautorefname}{Appendice}
\renewcommand*{\equationautorefname}{Équation}
\renewcommand*{\itemautorefname}{Point}
}
\providecommand{\subfigureautorefname}{\figureautorefname} % Fix to getting autorefs for subfigures right
}{\relax}
\makeatother

%----------------------------------------------------------------------------------------

\usepackage{classicthesis} 

%----------------------------------------------------------------------------------------
%	CHANGING TEXT AREA 
%----------------------------------------------------------------------------------------

%\areaset[current]{312pt}{761pt} % 686 (factor 2.2) + 33 head + 42 head \the\footskip
%\setlength{\marginparwidth}{7em}%
%\setlength{\marginparsep}{2em}%

%----------------------------------------------------------------------------------------
%	USING DIFFERENT FONTS
%----------------------------------------------------------------------------------------

%\usepackage{lmodern} % <-- no osf support :-( 
%\usepackage{palatino}
%\linespread{1.05} % a bit more for Palatino
\usepackage[frenchstyle, narrowiints, partialup]{kpfonts} % oldstyle notextcomp
%\mathversion{sf}
%\usepackage[osf]{libertine}
%\usepackage{hfoldsty} % Computer Modern with osf
%\usepackage[light,condensed,math]{iwona} 
%\renewcommand{\sfdefault}{iwona}

%---------------------------
% loading math options
%---------------------------
\makeatletter
\@ifpackageloaded{amsmath}{%%%%%%%%%%%%%%%%%%%%%%%%%%%%%%%%%%%%
% Math style configuration page - v2.1
% By pavel. Last edited July 16 2015
% Tribute to Igor Kortchemski
%%%%%%%%%%%%%%%%%%%%%%%%%%%%%%%%%%%%

\usepackage{amsthm}
\usepackage{amssymb}
\usepackage{xparse}
\usepackage{thmtools}

%language options

\renewcommand{\listtheoremname}{Table des théorèmes}
%shortcuts
\newcommand{\R}{\mathbb{R}}
\newcommand{\C}{\mathbb{C}}
\newcommand{\Z}{\mathbb{Z}}
\newcommand{\N}{\mathbb{N}}
\newcommand{\Q}{\mathbb{Q}}
\newcommand{\fii}{\varphi}
\newcommand{\In}{\mathcal{I}}
\newcommand{\surj}{\twoheadrightarrow}
\newcommand{\inj}{\hookrightarrow}
\newcommand{\bij}{\leftrightarrow}
\newcommand{\dd}{\mathrm{d}}
\newcommand{\db}{\overline{\partial}_b}
\newcommand{\eitheta}{e^{i\theta}}

%operators
\DeclareMathOperator{\Hess}{Hess}
\DeclareMathOperator{\Vois}{Vois}
\DeclareMathOperator{\supp}{supp}
\DeclareMathOperator{\diag}{diag}



% theorems configuration

\makeatletter
\newtheoremstyle{indented}
{7pt} %vertical space before
{7pt} % vertical space after
{\addtolength{\@totalleftmargin}{2.5em}
	\addtolength{\linewidth}{-3.5em}
	\parshape 1 3.5em \linewidth} %body font
{} %indent
{\bfseries} %header font
{.} %punctuation
{.5em} %horizontal space after header
{} %header specification

\newcounter{exo}
\newenvironment{exo}[1][]
{\edef\@exer{#1}
\ifx\@exer\empty{}\else{\setcounter{exo}{#1}\addtocounter{exo}{-1}}\fi\refstepcounter{exo}
\noindent{\bf Exercice \theexo.}
\hspace{0,5em}}
{\medskip\par}

\newcounter{sol}
\newenvironment{sol}[1][]
{\edef\@exer{#1}
\ifx\@exer\empty{}\else{\setcounter{sol}{#1}\addtocounter{sol}{-1}}\fi
\refstepcounter{sol}
\noindent {\it \underline{Solution de l'exercice \thesol.}
\hspace{0.5em}}}
{\medskip\par}
\makeatother

\theoremstyle{indented}

\newtheorem{defn}{Définition}[chapter]

\declaretheorem[name=Théorème]{theorem}
%\newtheorem{theoremaux}{Théorème}   %theorems have their own numerotation and are accessible through the TOC
%\NewDocumentEnvironment{theorem}{o}  %TOC accessibility
 % {\IfNoValueTF{#1}
    %{\theoremaux\addcontentsline{toc}{subsection}{\protect Théorème \thetheoremaux}}
   % {\theoremaux[#1]\addcontentsline{toc}{subsection}{\protect\kern -1em Théorème \thetheoremaux\enspace(#1)}}
    %\ignorespaces}
  %{\endtheoremaux}
\newenvironment{preuve}{
\noindent \textbf{Démonstration. }}{\hfill $\square$\par}

\newtheorem{exemple}[defn]{Exemple}
\newtheorem{prop}[defn]{Proposition}
\newtheorem{corr}[defn]{Corollaire}
\newtheorem{por}[defn]{Porisme}
\newtheorem{ex}[defn]{Exemple}
\newtheorem{lem}[defn]{Lemme}
\newtheorem{ax}{Axiome}  %Axioms have their own numerotation

\theoremstyle{definition}
\newtheorem{rem}[defn]{Remarque} %remarks are not indented
\newtheorem{rems}[defn]{Remarques}


%--------------
% Mise en page mathématique
%--------------
\addtolength{\jot}{.2em}}{}
\makeatother

\newcommand{\chantier}{\begin{center}
		\includegraphics[width=3cm]{gfx/travaux}
	\end{center}}